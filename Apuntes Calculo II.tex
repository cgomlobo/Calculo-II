\documentclass[12pt, titlepage]{article}
\usepackage[utf8]{inputenc}
\usepackage{amsfonts}
\usepackage{xfrac}
\usepackage{tikz}
\usepackage{amsmath}
\usepackage{mathtools}
\usepackage{cancel}
\usepackage{graphicx}
\usepackage{accents}
\def\checkmark{\tikz\fill[scale=0.4](0,.35) -- (.25,0) -- (1,.7) -- (.25,.15) -- cycle;}
\newcommand{\eqc}[1]{\stackrel{\mathclap{\normalfont\mbox{\scriptsize{#1}}}}{=}}
\newcommand{\leqc}[1]{\stackrel{\mathclap{\normalfont\mbox{\scriptsize{#1}}}}{\leq}}
\newcommand{\geqc}[1]{\stackrel{\mathclap{\normalfont\mbox{\scriptsize{#1}}}}{\geq}}
\newcommand{\inc}[1]{\stackrel{\mathclap{\normalfont\mbox{\scriptsize{#1}}}}{\in}}
\newcommand{\R}{\mathbb{R}}
\newcommand{\N}{\mathbb{N}}
\newcommand{\bfrac}[2]{\scalebox{1.4}{$\frac{#1}{#2}$}}
\newcommand{\spac}{\, \, \,}

\begin{document}

\title {Cálculo II}
\author {Carlos Gómez-Lobo Hernaiz}
\date{}
\maketitle

%\TEMA 1

\part* {Tema 1. Intrducción al espacio de varias variables $\displaystyle \mathbf{\mathbb{R}^n}$}

\section {Vectores, producto escalar y distancia}

\begin{itemize}

\item En $\displaystyle \mathbb{R}^2$ los puntos, $p \in \displaystyle \mathbb{R}^2$, se representan mediante pares ordenados de números reales, $p=(a, b)$. \\$\mathbb{R}^2 = \{(x, y):x, y \in \mathbb{R}\}$.

\item En $\displaystyle \mathbb{R}^3$ los puntos, $p \in\displaystyle  \mathbb{R}^3$, se representan mediante ternas ordenadas de números reales, $p=(a, b, c)$. \\$\displaystyle \mathbb{R}^3 = \{(x, y, z):x, y, z \in \displaystyle \mathbb{R}\}$.

\item En $\mathbb{R}^n$ los puntos, $p \in \displaystyle \mathbb{R}^n$, se representan mediante pares ordenados de números reales, $p=(x_1, x_2, ..., x_n)$.\\ $\displaystyle \mathbb{R}^n = \{(x_1, x_2, ..., x_n):x_1 \in \displaystyle \mathbb{R}, i=1, 2, ..., n\}$.

\end{itemize}

\noindent -\underline{Definición:} Geométricamente, un \textbf{vector} se define como un segmento de recta "orientado" con punto inicial en el origen de coordenadas y punto final en un punto de $\displaystyle \mathbb{R}^n$.

\vspace{3mm}

\noindent *\underline{Observación:} Si n=3, la notación habitual parala base canónica es: \\$\vec{i}=(1, 0, 0), \vec{j}=(0, 1, 0), \vec{k}=(0, 0, 1)$.

%Operaciones en Rn

\subsection{Operaciones entre elementos de $\mathbf{\displaystyle \mathbb{R}^n : \vec{x}=(x_1, x_2, ..., x_n),} \\ \mathbf{ \vec{y}=(y_1, y_2, ... y_n)}$}

\begin {enumerate}

\item Suma: $\vec{x}+\vec{y}=(x_1+y_1, x_2+y_2, ..., x_n+y_n)$

\item Multiplicación por un escalar, $\lambda \in \mathbb{R}: \lambda \vec{x}=(\lambda x_1, \lambda x_2, ..., \lambda x_n)$

\item Producto escalar: $<\quad ,\quad > : \displaystyle \mathbb{R}^n \times \displaystyle \mathbb{R}^n \rightarrow <\vec{x}, \vec{y}>\displaystyle \eqc{not} \vec{x} \cdot \vec{y}= \\
=  x_1y_1+x_2+y_2+...+x_ny_n = \displaystyle \sum_{i=1}^n x_i\,y_i$

\end {enumerate}

%Propiedades

\subsubsection*{Propiedades}
 Sean $\vec{x}, \vec{y}, \vec{z} \quad y \quad \lambda , \beta \in \displaystyle \mathbb{R}$

\begin{enumerate}

\item Linealidad: $(\lambda \, \vec{x} + \beta \, \vec{y}) \cdot \vec{z} = \lambda \, \vec{x} \, \vec{z} + \beta \, \vec{y} \, \vec{z}$

\item Conmutativa: $<\vec{x}, \vec{y}>=<\vec{y}, \vec{x}>$

\item $< \vec{x}, \vec{x} > \geq 0  \, \, \mathrm{y} < \vec{x}, \vec{x} > = 0 \iff \vec{x} = \vec{0} = (0, 0,..., 0)$

\end{enumerate}

%La norma euclídea

\noindent -\underline{Definición:} La \textbf{norma euclídea} de un vector $\vec{x} \in \displaystyle \mathbb{R}^n$(y se escribe $\|\vec{x}\| \mathrm{)}$ como: $\|\vec{x}\| = \sqrt{<\vec{x}, \vec{x}>} = \Big{(}\displaystyle \sum_{i = 1}^{n} x_i^{\,2}\Big{)}^{^1/_2}$.

\vspace{3mm}

*\underline{Observación:} Sea $\vec{x} \in \displaystyle \mathbb{R}^n$ tal que $\|\vec{x}\| = 1$, se dice que es \textbf{unitario}. Sea  $\vec{x} \in \displaystyle \mathbb{R}^n$ un vector no unitario, podemos \textbf{normalizarlo}:\\
{\large$\frac{\vec{x}}{\|\vec{x}\|} \rightarrow$} vector unitario.

\subsubsection*{Significado geométrico de la norma euclídea}

La norma $\|\vec{x}\|$ representa la \textbf{longitud} del vector $\vec{x}$. Not.: $long(\vec{x}) = |x| = d(0, x)$.

%Desigualdad de Cauchy-Schwarz

\subsubsection*{Desigualdad de Cauchy-Schwarz en $\mathbf{\mathbb{R}^n}$}

Sean $\vec{x}, \vec{y} \in \displaystyle \mathbb{R}^n$\\  $$\boxed{|<\vec{x}, \vec{y}>| \leq \|\vec{x}\| \cdot \|\vec{y}\| = \bigg{|}\displaystyle \sum_{i=1}^n x_i \, y_i \bigg{|} \leq \bigg{(}  \sum_{i=1}^n x_i^2 \bigg{)}^{\sfrac{1}{2}} \cdot \bigg{(} \sum_{i=1}^n y_i^2 \bigg{)}^{\sfrac{1}{2}}}$$

\vspace{3mm}

\noindent\underline{Demostración:} Tomamos $a= \vec{y} \cdot \vec{y} = \|\vec{y}\|^2 \quad \mathrm{y} \quad b = -\vec{x} \cdot \vec{y} \,$ y aplicamos propiedades del producto escalar al vector $a  \vec{x} + b \vec{y}$.

\vspace{3mm}

-Si $a=0 \Rightarrow \vec{y} = 0$  \checkmark

\vspace{1mm}

-Si $a \neq 0 \Rightarrow 0 \leq ( a\vec{x} + b \vec{y} ) (a \vec{x} + b \vec{y} ) = a^2 \vec{x}\, \vec{x} + 2ab \vec{x} \vec{y} + b^2 \vec{y} \vec{y} = \\ = \|\vec{y}\|^4 \cdot \|\vec{x}\|^2-2 \, \|\vec{y}\|^2 \, ( \vec{x} \cdot \vec{y} )^2 + ( -\vec{x} \cdot \vec{y} )^2 \cdot \|\vec{y} \|^2 = \|\vec{y} \|^4 \cdot \|\vec{x}\|^2 - \|\vec{y}\|^2 \cdot ( \vec{x} \cdot \vec{y} )^2 \Rightarrow \\ \Rightarrow 0 \leq \|\vec{y}\|^2 \cdot \|\vec{x}\|^2 - ( \vec{x} \cdot \vec{y} )^2 \Rightarrow ( \vec{x} \cdot \vec{y} )^2 \leq \|\vec{x}\|^2 \cdot \|\vec{y}\|^2 \Rightarrow |\vec{x} \cdot \vec{y} |\leq \|\vec{x}\| \cdot \|\vec{y}\| \, \checkmark \\
-\|\vec{x}\| \cdot \|\vec{y}\| \leq < \vec{x}, \vec{y} > \leq \|\vec{x}\| \cdot \|\vec{y}\| \Rightarrow -1 \leq \bfrac{<\vec{x}, \vec{y}>}{\|\vec{x}\| \cdot \|\vec{y}\|} \leq 1 \Rightarrow \\ \Rightarrow \cos{\theta} \eqc{def}  \bfrac{\vec{x} \cdot \vec{y}}{\|\vec{x}\| \cdot \|\vec{y}\|} \iff < \vec{x}, \vec{y} > = \|\vec{x}\| \, \|\vec{y}\| \, \cos{\theta}$

\vspace{3mm}

\noindent-\underline{Definición:} Dos vectores $\vec{x}, \vec{y} \in \mathbb{R}^n$ son \textbf{perpendiculares/ortogonales} si y solo si $< \vec{x}, \vec{y}> = 0$. Not.: $\vec{x} \perp \vec{y}$.

%Teorema de Pitágoras

\subsubsection*{Teorema de Pitágoras en $\mathbf{\mathbb{R}^n}$}

Sean $\vec{x}, \vec{y} \in \mathbb{R}^n$ tal que $\vec{x} \perp \vec{y}$, entonces: 
$$\boxed{\|\vec{x} + \vec{y}\|^2 = \|\vec{x}\|^2 + \|\vec{y}\|^2}$$

\noindent\underline{Demostración:} $\|\vec{x} + \vec{y}\|^2 = (\vec{x}+ \vec{y}) (\vec{x}+ \vec{y}) = \vec{x}  \vec{x} + \cancelto{0}{2 \vec{x} \vec{y}}+ \vec{y} \vec{y} = \|\vec{x}\|^2 + \|\vec{y}\|^2 \, \checkmark$

\vspace{3mm}

%Teorema del coseno

\subsubsection*{Teorema del coseno}

Sean $\vec{x}, \vec{y} \in \mathbb{R}^n$ vectores cualesquiera, entonces:
$$\boxed{\|\vec{x} - \vec{y}\|^2 = \|\vec{x}\|^2 + \|\vec{y}\|^2 - 2 \, \|\vec{x}\| \, \|\vec{y}\| \, \cos{\theta}}$$
\underline{Demostración:} $\|\vec{x} - \vec{y}\|^2 = (\vec{x} - \vec{y}) (\vec{x} - \vec{y}) = \vec{x} \vec{x} - 2 \vec{x} \vec{y} + \vec{y} \vec{y} = \vspace{1mm} \\ 
=\|\vec{x}\|^2 + \|\vec{y}\|^2 - 2 \, \|\vec{x}\| \, \|\vec{y}\| \, \cos{\theta} \, \checkmark$

\vspace{3mm}

%Propiedades de la norma euclídea

\subsubsection*{Propiedades de la norma euclídea}
Sea $\vec{x} \in \mathbb{R}^n$ y $\lambda \in \mathbb{R}$: \\ 

$(i) \|\vec{x}\| \geq 0$ y $ \|\vec{x}\| = 0 \iff \vec{x} = \vec{0}$ \vspace{1mm}

$(ii) \| \lambda \vec{x}\| = |\lambda| \cdot \|\vec{x}\|$ \vspace{1mm}

$(iii)$Desigualdad tiangular: $\|\vec{x} + \vec{y}\| \leq \|\vec{x}\| + \|\vec{y}\|$ \vspace{1mm}

$(iv)$Desigualdad triangular al revés: $\big{|} \|\vec{x}\| - \|\vec{y}\|\big{|} \leq \|\vec{x}- \vec{y}\|$
\vspace{3mm}

\noindent\underline{Demostraciones:}\vspace{5mm}

$(i) \, \|\vec{x}\| = \sqrt{x_1^2 + x_2^2+...+ x_n^2} \geq 0$ y $ \|\vec{x}\| = \sqrt{x_1^2 + x_2^2 + ...+ x_n^2} = 0 \iff  \indent \iff x_1, x_2,..., x_n = 0 \Rightarrow \vec{x} = \vec{0}$

$(ii) \, \displaystyle \|\lambda \vec{x}\| = \Big{(} \sum_{i=1}^n \lambda ^2 \, x_i^2 \Big{)}^{\sfrac{1}{2}} = \sqrt{\lambda^2} \, \Big{(} \sum_{i=1}^n x_i^2 \Big{)}^{\sfrac{1}{2}} = |\lambda| \, \|\vec{x}\|$

$(iii) \|\vec{x} + \vec{y}\|^2 = (\vec{x} + \vec{y})(\vec{x} + \vec{y}) = \vec{x} \vec{x} + 2 \vec{x} \vec{y} + \vec{y} \vec{y} \leqc{C-S} \|\vec{x}\|^2 + 2 \, \|\vec{x}\| \, \|\vec{y}\| + \|\vec{y}\|^2 = \indent = (\|\vec{x}\| + \|\vec{y}\|)^2 \Rightarrow \|\vec{x} + \vec{y}\| \leq \|\vec{x}\| + \|\vec{y}\|$ \vspace{5mm}

\noindent-\underline{Definición:} Una \textbf{distancia/métrica} es una función $d:\mathbb{R}^n \times \mathbb{R}^n \rightarrow \mathbb{R}^+$ tal que cumple: \vspace{2mm}

(1) $ d(x, y) = 0 \iff x=y$ \vspace{1mm}

(2) $ d(x, y) = d(y, x)$ \vspace{1mm}

(3) $d(x, z) \leq d(x, y) + d(y, z)$ \\

\noindent-\underline{Definición:} La \textbf{distancia euclídea} entre dos puntos $x = (x_1, x_2,..., x_n)$ e $y = (y_1, y_2,..., y_n)$ de $\mathbb{R}^n$ se define como:
$$\boxed{d(x, y) = \sqrt{(x_1-y_1)^2+(x_2+y_2)^2+...+(x_n+y_n)^2}=\bigg{(} \sum_{i=1}^n (x_i - y_i)^2 \bigg{)}^{\sfrac{1}{2}} = \|\vec{x} - \vec{y}\|}$$\break
(1) $\|\vec{x} - \vec{y}\| = 0 \iff \vec{x} = \vec{y}$\\
(2) $\|\vec{x} - \vec{y}\| = \|\vec{y} - \vec{x}\|$ \\
(3) $\|\vec{x} - \vec{z}\| \leq \|\vec{x} - \vec{y}\| + \|\vec{y} - \vec{z}\|$

%CONCEPTOS MÉTRICOS EN EL ESPACIO EUCLÍDEO

\section{Conceptos métricos en el espacio euclídeo en $\mathbb{R}^{\mathbf{n}}$}
\vspace{3mm}

%BOLAS ABIERTAS Y CONJUNTOS ABIERTOS EN Rn

\subsection{Bolas abiertas y conjuntos abiertos en $\mathbb{R}^{\mathbf{n}}$}
\vspace{3mm}

-\underline{Definiciones:}

\vspace{5mm}
\noindent
(1) Dado un punto $x_0 \in \mathbb{R}^n$ y un número real $r>0$, llamaremos \textbf{bola abierta} de centro $x_0$ y radio $r$ (y escribiremos $B_r (x_0) $ o $B(x_0, r)$al conjunto:
$$\boxed{B_r (x_0)  \eqc{not} B(x_0, r) = \{ x \in \mathbb{R}^n : \|x-x_0\|<r\}}$$
\vspace{1mm}

\noindent
$n=1 \rightarrow x_o \in \mathbb{R} \quad -r<x-x_o<r \\ B_r(x_0) = \{ x \in \mathbb{R} : |x-x_0|<r\} = \{ x \in \mathbb{R} : x \in (x_0-r, x_0+r) \}$

\vspace{5mm}
\noindent
$n=2 \rightarrow (x_0, y_0) \in \mathbb{R}^2 \\ B_r (x_0, y_0) = \{ (x, y) \in \mathbb{R}^2 : \underbrace{\|(x-y) - (x_0 - y_0) \|}_{(x-x_0)^2 + (y - y_0)^2} < r\}$

\noindent
$n=3 \rightarrow (x_0, y_0, z_0) \in \mathbb{R}^3 \\
B_r (x_0, y_0, z_0) = \{ (x, y, z) \in \mathbb{R}^3 : (x-x_0)^2 + (y - y_0)^2 + (z - z_0)^2 < r$

\vspace{5mm}
\noindent
(2) Decimos que $A \in \mathbb{R}^n$ es un \textbf{conjunto abierto} si $\forall x_0 \in A \, \, \exists r>0$ tal que $B_r (x_0) \subset A$.

\vspace{5mm}
\noindent
(3) Un \textbf{entorno} del punto $x \in \mathbb{R}^n$ es un conjunto abierto que lo contiene.

%Proposiciones

\begin{itemize}

\item \underline{Proposición 1:} Si $x_0 \in \mathbb{R}^n$ y $r>0 \Rightarrow$ la bola abierta $B_r(x_0)$ es un conjunto abierto.

\quad\underline{Demostración:} $\forall x \in B_r(x_0) \, \exists s \in \R , s>0$ tal que $B_s(x) \subset B_r(x_0)$.

Basta con tomar $s = r - \|x-x_0\|$, entonces $\forall y \in B_s(x) \quad \|y - x_0\| = \\
 = \|y -x +x -x_0\| \leq \|y-x\| + \|x - x_0\| < s + \|x - x_0\| = r - \|x - x_0\| + \|x - x_0\| = r$
\vspace{2mm}

\quad\underline{Ejemplo:} Demostrar que $A = \{ (x, y) \in \R^2 : y>0\}$ es un conjunto abierto.

$\forall (x_0, y_0) \in A \spac \exists r : B_r(x_0, y_0) \subset A$, es decir, $y_0 > 0$\\
Basta con tomar $r = y_0 > 0$.\\
$\forall (a, b) \in B_r(x_0, y_0) \Rightarrow$ ¿$(a, b) \in A$, es decir, $b>0$?\\
Como $ (a, b) \in B_r(x_0, y_0) \Rightarrow \|(a, b) - (x_0, y_0)\| < r = y_0 \Rightarrow \\
\Rightarrow (a - x_0)^2 + (b - y_0)^2 < y_0^2 \Rightarrow (b - y_0)^2 < y_0 ^2 \Rightarrow -y_0 < b-y_0 < y_0 \Rightarrow \\
 \Rightarrow 0< b < 2 y_0 \Rightarrow b > 0 \Rightarrow (a, b) \in A$

\item \underline{Proposición 2:} La unión (finita o infinita) de conjuntos abiertos es un conjunto abierto.

\item \underline{Proposición 3:} La intersección finita de abiertos es un conjunto abierto.\vspace{1mm}

\noindent *\underline{Observación:} La intersección finita de abiertos no es, en general, un abierto.

\underline{Demostración:} $\{A_1, A_2,..., A_n\}$ conjuntos abiertos y $A = \displaystyle \bigcap_{i = 1}^n A_i$.  ¿A es un conjunto abierto? 

Dado $x \in A \Rightarrow x \in A_1, x \in A_2,..., x \in A_n \Rightarrow \exists r_1, r_2,..., r_n$ tal que $ B_{r_1}(x) \subset  A_1, B_{r_2}(x) \subset A_2,..., B_{r_n}(x) \subset A_n$, tomando $r \leq min \{r_i\}^{n_i=1} \Rightarrow B_r(x) \subset \displaystyle \bigcap_{i = 1}^n A_i = A$.

\end{itemize}

%LIMITES DE SUCESIONES EN Rn. COMPLETITUD.

\subsection{Límites de sucesiones en $\mathbf{\R^n}$. Completitud.}
\vspace{3mm}

-\underline{Definición:} Una \textbf{sucesión en} $\mathbf{\R^n}$ es una colección de n sucesiones en $\R : \{A_k\}_{k \in \N} = \{(a_1^k, a_2^k,..., a_n^k)\}_{k \in \N}$.
\vspace{1mm}

\indent \underline{Ejemplos:} $\{A_k\}_{k \in \N} = \left\{ \left (k, \bfrac{1}{k^2}, e^{-k}\right ) \right \}_{k \in \N}; \{A_k\}_{k \in \N} = \left \{ \left (\bfrac{3k^2}{k^2-1}, \bfrac{1}{k}, \log \bfrac{k+1}{k} \right )\right \}_{k \in \N}$.\vspace{5mm}

\noindent-\underline{Definición:} Dada una sucesión $\{A_k\}_{k \in \N} \in \R^n$ y un punto $L \in \R^n$, diremos que el límite de $\{A_k\}_{k \in \N}$ es $L$ y escribimos: 
\[
\boxed{\displaystyle \lim_{k \to \infty} A_k = L \iff \forall \epsilon > 0 \spac \exists N \in \N : \|A_k - L\| < \epsilon \spac \forall k > N\text{, es decir, } \forall k > N \spac A_k \in B_\epsilon (l)}
\]\vspace{2mm}

%LEMA

\noindent $\bullet$ \underline{Lema:} Dada una sucesión en $\R^n : \{A_k\}_{k \in \N} = \{(a^k_1, a^k_2,..., a^k_n)\}_{k \in \N}$ y un punto $L = (l_1, l_2,..., l_n) \in \R^n$, se tiene:
\[
\boxed{\displaystyle \lim_{k \to \infty} A_k = L \iff \lim_{k \to \infty} a_i^k = l_i \spac \forall i = 1, 2,..., n}
\]

\underline{Demostración:}
$\Rightarrow$) Queremos ver que $\forall \epsilon > 0 \spac \exists N \in \N : |a_i^k - l_i| < \epsilon \spac \forall k > N \forall i = 1, 2,..., n$.\vspace{1mm}\\
$|a_i^k - l_i| =\sqrt{(a_i^k - l_i)^2} \leq \sqrt{(a_1^k - l_1)^2 + (a_1^k - l_2)^2 + ... + (a_i^k - l_i)^2 +... + (a_n - l_n)^2} = \quad = \|A_k - L\| \leq \epsilon \spac \checkmark$\vspace{3mm}

\noindent $\Leftarrow$) Como $\displaystyle \lim_{k \to \infty} a_i^k = l \spac \forall i = 1, 2,..., n$, sabemos que $\forall \epsilon > 0 :$

$\exists N_1 \in \N \text{ tal que } |a_1^k - l_1| < \bfrac{\epsilon}{\sqrt{n}} \spac \forall k > N_1$

$\exists N_2 \in \N \text{ tal que } |a_2^k - l_2| < \bfrac{\epsilon}{\sqrt{n}} \spac \forall k > N_2$

\quad \quad \quad \quad \quad \quad \quad \quad \vdots

$\exists N_n \in \N \text{ tal que } |a_n^k - l_n| < \bfrac{\epsilon}{\sqrt{n}} \spac \forall k > N_n$
\vspace{1mm}

\noindent Entonces, $\forall k \geq \max \{N_1, N_2,..., N_n\}$ tenemos simultáneamente que $|a_i^k - l_k| < \bfrac{\epsilon}{\sqrt{n}} \spac \forall i = 1, 2,..., n$, por tanto:
\vspace{1mm}

\noindent$\|A_k - L\| = \sqrt{(a_1^k - l_1)^2 + (a_1^k - l_2)^2 + ... + (a_i^k - l_i)^2 +... + (a_n - l_n)^2} \leq \vspace{1mm} \\ \leq \sqrt{n \left (\bfrac{\epsilon}{\sqrt{n}}\right )^2} = \epsilon \spac \checkmark$

\underline{Ejemplo:} Hallar el límite $\displaystyle \lim_{k \to \infty} \left \{\left ( \bfrac{\log{x}}{x^2}, \bfrac{x^2-1}{5 x^2} \right )\right \}_{k \in \N}$.

\[
\left.\begin{array}{lc}\displaystyle

\lim_{k \to \infty} \bfrac{\log{k}}{k^2} \eqc{L'H} \lim_{k \to \infty} \bfrac{\sfrac{1}{k}}{2 k} = \lim_{k \to \infty} \bfrac{1}{2 k^2} = 0 \\

\lim_{k \to \infty} \bfrac{x^2 - 1}{5 x^2} = \bfrac{1}{5}

\end{array}\right \}
\displaystyle \lim_{k \to \infty} \left \{\left ( \bfrac{\log{x}}{x^2}, \bfrac{x^2-1}{5 x^2} \right )\right \} = (0, 5)
\]
\vspace{3mm}

%Propiedad de completitud

\noindent-\underline{Definición:} La \textbf{propiedad de completitud} dice que toda sucesión de Cauchy converge en el espacio.
\vspace{5mm}

\noindent-\underline{Definición:} Una sucesión $\{ A_k\}_{k \in \N}$ es de \textbf{Cauchy} en $\R^n$ si $\forall \epsilon >0 \spac \exists N \in \N : \|A_{k_1} - A-{k_2}\| < \epsilon \spac \forall k_1, k_2 > N$.
\vspace{5mm}

\noindent $\bullet$ \underline{Teorema:} El espacio euclídeo $\R^n$ es \textbf{completo}, es decir, toda sucesión de Cauchy es convergente.
\vspace{2mm}

\underline{Demostrción:} Pendiente como ejercicio.
\vspace{5mm}

\subsection{Más sobre la topología de $\mathbf{\R^n}$}
\vspace{5mm}

%Punto interior

-\underline{Definición:} Sea $A$ un conjunto de $\R^n$. Diremos que un punto $x \in A$  es un \textbf{punto interior} de $A$ si $\exists r > 0 \text{ tal que } B_r (x) \in A$. Al conjunto de puntos interiores de $A$ se le denota $\mathring{A} \eqc{not} int(A)$.
\vspace{3mm}

\underline{Ejercicio 1:} Dado $A = \{(x, y) \in \R^n : |x|<1, |y| \leq 1 \}$. Hallar $\mathring{A}$.
\vspace{3mm}

1) $\forall (x, y) \N A : |y| = 1 \Rightarrow (x, \pm 1) (\text{ con } |x| < 1) \spac (x, 1) \in A, \forall r > 0 \rightarrow \\ \rightarrow (x, 1 + \sfrac{r}{2}) \inc{?} A \spac |1 + \sfrac{r}{2}| = 1 + \sfrac{r}{2} > 1 \Rightarrow B_r(x, 1) \not \subset A$

2) $\forall (x, y) \subset A \spac |x|, |y| < 1$:
\[
*\left. \begin{array}{rc}

r \leq 1- |x| donde |y| \leq |x|\\
r \leq 1- |y| donde |x| \leq |y|

\end{array}\right \}\spac\parbox{10cm}{en cualquier caso $r = 1 - \max{\{|x|, |y|\}} = \\ =\min{\{1 - |x|,
 1 - |y|\}}$}
\]

Tenemos que ver que $B_r (x, y) \subset A$, es decir, $\forall (a, b) \in B_r(x, y)$ se tiene que $|a| < 1
 \text{ y } |b| \leq 1 \Rightarrow (a, b) \in A$

$|a| = |a - x + x| \leqc{D.T.} \underbrace{|a-x|}_{\sqrt{(a - x)^2}} + |x| \leq \underbrace{\|(a, b) - (x, y
)\|}_{\sqrt{(a -x)^2 + (b - y)^2}} + |x| < r + |x| \leqc{*} \\ \leqc{*} 1 - |x| +|x| =1$
\vspace{3mm}

$|b| = |b - y + y| \leqc{D.T.} |b - y| +|y| < r + |y| \leq 1 - |y| + |y| = 1$
\vspace{3mm}

$\forall (a, b) \in B_r(x, y)$ se tiene que $(a, b) \in A \, \checkmark$
\vspace{5mm}

%Propiedades

\noindent$\bullet$ \underline{Propiedades:}
\vspace{3mm}

(1) $\mathring{A}$ es un conjunto abierto.
\vspace{3mm}

(2) $\mathring{A} \subset A$.
\vspace{3mm}

(3) Si $A$ es un conjunto abierto, entonces $\mathring{A} = A$.
\vspace{5mm}

%Conjunto cerrado

\noindent-\underline{Definición:} Diremos que un conjunto $C \subset \R^n$ es un \textbf{conjunto
 cerrado} si su complementario $C^c \subset \R^n$, es un conjunto abierto.
\vspace{3mm}

\underline{Ejercicio 2:} Demostrar que la bola $C = \{ (x, y) : \underbrace{x^2 + y^2 \leq R^2}_{\|(x, y
)\|} \leq R\}$ es un conjunto cerrado en $\R^2$.
\vspace{3mm}

$C$  cerrado en $\R^2 \text{ si } C^c = \{(x, y) \in \R^2 : \underbrace{x^2 + y^2}_{\|(x, y)\|} > R\}$ es
 un conjunto abierto en $\R^2$, es decir, $\forall (x_0, y_0) \in C^c \spac \exists r > 0 : B_r (x_0, y_0)
 \subset C^c$.
\vspace{3mm}

Tomamos $r = \| (x_0, y_0) \| - R \Rightarrow \forall (a, b) \in B_r (x_0, y_0) \text{, ¿} a, b \in C^c \text{,
 es decir, } \| (a, b) \| > \quad > R$?
\vspace{2mm}

$\| (x_0, y_0) \| = \| (x_0, y_0) - (a, b) + (a, b) \| \leqc{D.T.} \underbrace{\| (x_0, y_0) - (a, b) \|}_{< r =
 \| (x_0, y_0) \| - R} + \| (a, b) \| < \quad < \| (x_0, y_0) \| - R + \| (a, b) \| \Rightarrow \| (x_0, y_0) \| <
 \| (x_0, y_0) \| - R + \| (a, b) \| \Rightarrow \\ \Rightarrow R < \| (a, b) \| \, \checkmark$
\vspace{5mm}

\noindent$\bullet$ \underline{Proposición 1:}
\vspace{3mm}

(i) La unión finita de conjuntos cerrados es un conjunto cerrado.
\vspace{3mm}

(ii) La intersección (infinita o finita) de conjuntos cerrados, es un conjunto \indent cerrado.
\vspace{3mm}

$\ast$\underline{Observación:} La unión infinita de conjuntos cerrados no es, en general, \indent un
 conjunto cerrado.
\vspace{3mm}

-Bolas \textbf{abiertas} en $\R^2: \spac B_R (x_0, y_0) = \{ (x, y) \in \R^2 : \underbrace{(x-x_0)^2 + (y -
 y_0)^2 < R^2}_{\| (x, y) - (x_0, y_0)\| <R}\}$

-Bolas \textbf{cerradas} en $\R^2:  \spac \overline{B_R (x_0, y_0)} = \{(x , y) \in \R^2 : \underbrace{(x -
 x_0)^2 + (y - y_0)^2 \leq R^2}_{\|(x, y) - (x_0, y_0)\| \leq R}\}$
\vspace{3mm}

\noindent$\ast$\underline{Observación 1:} No todos los conjuntos de $\R^n$ son necesariamente abiertos 
o cerrados.
\vspace{3mm}

\noindent$\ast$\underline{Observación 2:} El vacío y $\R^n$ (por convención) son abiertos y cerrados al
 mismo tiempo.
\vspace{3mm}

\noindent$\bullet$ \underline{Proposición 2 (Caracterización de los conjuntos cerrados por sucesiones):}

\[
\boxed{C \subset \R^n \text{ es un cerrado } \iff \forall {A_k}_{k \in \N} \subset C \text{ con }
 \displaystyle \lim_{k \to \infty}^{} A_k = L \text{ se tiene que } \\ L \in C}
\]

\underline{Demostrción:} Argumentamos (en ambas implicaciones) por reducción al absurdo:
\vspace{2mm}

$\Rightarrow$ ) Supongamos que $L \not\in C \Rightarrow L \in C^c \text{ abierto } \Rightarrow \exists 
r > 0 \text{ tal que } B_r (L) \subset C^c \text{. Como} \displaystyle \lim_{k \to \infty} A_k = L, \spac
\forall \epsilon > 0  \exists N \in \N \text{ tal que }|A_k - L| > \epsilon \spac \forall k > N \text{, es decir, 
}A_k \in B_\epsilon (L) \spac \forall k>N \text{. Si tomamos } \epsilon = r : A_k \in B_r (L) \text{
 \underline{Contradicción}}\spac A_k \in C \text{ y } \\ B_r (L) \subset C^c$
\vspace{3mm}

$\Leftarrow$ ) Supongamos que $C$ no es un cerrado $\Rightarrow C^c$ no es un abierto $\Rightarrow \\
\Rightarrow \exists x_0 \in C^c \text{ tal que } \forall r>0 \spac B_r (x) \not\subset C^c \text{, es decir, }
B_r (x_0) \cap C \neq \emptyset \spac  \forall{ r > 0 }\text{. Tomamos } \\ r =  \bfrac{1}{k}
 \Rightarrow B_{\sfrac{1}{k}} (x_0) \cap C \neq \emptyset \Rightarrow \exists \{A_k\}_{k \in \N} \subset
 B_{\sfrac{1}{k}} \cap C, \{A_k\}_{k \in \N} \rightarrow x_0 \in C \\ \text{\underline{Contradicción} porque } x_0 \subset C^c$.
\vspace{5mm}

%Punto de adherencia

\noindent-\underline{Definición:} Sea $C$ un conjunto cualquiera en $\R^n$ y $x \in \R^n$, diremos que 
$x$ es un \textbf{punto de adherencia/clausura} de $C$ si $\forall{r>0} \spac B_r (x) \cap C \neq \emptyset
$ y el conjunto de puntos que cumplen esta condición se denota:
\[
\boxed{\overline{C} = \{x \in \R^n : \forall{r>0} \spac B_r (x 
\cap C \neq \emptyset\}}
\]

\underline{Propiedades:} 
\vspace{3mm}

\indent \indent (1) $\overline{C}$ es un conjunto cerrado.
\vspace{3mm}

\indent \indent(2) $C \subset \overline{C}$.
\vspace{3mm}

\indent \indent(3) Si $C$ es cerrado $C = \overline{C}$.
\vspace{5mm}

\underline{Ejercicio 3:} Sea $C = \left \{ \left ( \bfrac{1}{k} , \bfrac{1}{k} \right ) \right \}_{k \in \N}
$probar que $C \in \overline{C}$.

$\displaystyle \lim_{k \to 0} \left (\bfrac{1}{k} , \bfrac{1}{k} \right ) = (0, 0) \Rightarrow \forall{\epsilon > 0} \exists N \text{ tal que  } \| \left (\bfrac{1}{k} , \bfrac{1}{k} \right ) - (0, 0) \| < \epsilon \\ \indent
\forall k > N. \text{ Basta tomar } r = \epsilon \Rightarrow \left (\bfrac{1}{k} , \bfrac{1}{k} \right ) \in
 B_\epsilon (0, 0) \forall{k \geq N} \Rightarrow \\ \indent \Rightarrow C \cap B_r (0, 0) \neq \emptyset$
\vspace{5mm}

%Punto de acumulación

\noindent-\underline{Definición:} Sea C un conjunto cualquiera de $\R^n$ y $x \in \R^n$. Diremos que $x$ es un \textbf{punto de acumulación} de $C$ si $\forall{r > 0} B_r(x) \setminus \{x\} \cap C \neq \emptyset 0$. A los puntos de acumulación se los denota por $C'$.
\vspace{3mm}

\underline{Propiedades:}
\vspace{3mm}

\indent \indent (1) $C' \subset \overline{C}$
\vspace{3mm}

\indent \indent (2) En general $\overline{C} \not\subset C'$

\end{document}