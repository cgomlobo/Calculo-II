\documentclass[10pt, titlepage]{article}
\usepackage[utf8]{inputenc}
\usepackage{amsfonts}
\usepackage{xfrac}
\usepackage{tikz}
\usepackage{amsmath}
\usepackage{mathtools}
\usepackage{cancel}
\usepackage{graphicx}
\usepackage{accents}
\usepackage{xcolor}
\usepackage{vmargin}
\setpapersize{A4}
\setmargins{2.5cm}       % margen izquierdo
{1.5cm}                        % margen superior
{16cm}                      % anchura del texto
{23.42cm}                    % altura del texto
{10pt}                           % altura de los encabezados
{1cm}                           % espacio entre el texto y los encabezados
{0pt}                             % altura del pie de página
{2cm}


\def\checkmark{\tikz\fill[scale=0.4](0,.35) -- (.25,0) -- (1,.7) -- (.25,.15) -- cycle;}
\newcommand{\eqc}[1]{\stackrel{\mathclap{\normalfont\mbox{\scriptsize{#1}}}}{=}}
\newcommand{\leqc}[1]{\stackrel{\mathclap{\normalfont\mbox{\scriptsize{#1}}}}{\leq}}
\newcommand{\geqc}[1]{\stackrel{\mathclap{\normalfont\mbox{\scriptsize{#1}}}}{\geq}}
\newcommand{\inc}[1]{\stackrel{\mathclap{\normalfont\mbox{\scriptsize{#1}}}}{\in}}
\newcommand{\gsc}[2]{\stackrel{\mathclap{\normalfont\mbox{\scriptsize{#2}}}}{#1}}
\newcommand{\R}{\mathbb{R}}
\newcommand{\N}{\mathbb{N}}
\newcommand{\bfrac}[2]{\scalebox{1.4}{$\frac{#1}{#2}$}}
\newcommand{\spac}{\, \, \,}
\newcommand{\definicion}{\noindent-\underline{Definición:} }
\newcommand{\teorema}[1][\!\!]{\noindent$\bullet$ \underline{Teorema #1:} }
\newcommand{\lema}[1][\!\!]{\noindent$\bullet$ \underline{Lema #1:} }
\newcommand{\proposicion}[1][\!\!]{\noindent$\bullet$ \underline{Proposición #1:} }
\newcommand{\observacion}[1][\!\!]{\noindent$\ast$ \underline{Observación #1:} }
\newcommand{\dindent}{\indent\indent}
\newcommand{\tindent}{\indent\indent\indent}
\newcommand{\limdef}[2]{$\forall \epsilon > 0 \spac \exists \delta > 0 : \|(x ,y) - #1\| < \delta \Rightarrow 
|f (x, y) - #2| < \epsilon$\vspace{3mm}

}

\begin{document}

\title {Cálculo II}
\author {Carlos Gómez-Lobo Hernaiz}
\date{}
\maketitle

%\TEMA 1

\part* {Tema 1. Intrducción al espacio de varias variables $\displaystyle \mathbf{\mathbb{R}^n}$}

\section {Vectores, producto escalar y distancia}

\begin{itemize}

\item En $\displaystyle \mathbb{R}^2$ los puntos, $p \in \displaystyle \mathbb{R}^2$, se representan mediante pares ordenados de números reales, $p=(a, b)$. \\$\mathbb{R}^2 = \{(x, y):x, y \in \mathbb{R}\}$.

\item En $\displaystyle \mathbb{R}^3$ los puntos, $p \in\displaystyle  \mathbb{R}^3$, se representan mediante ternas ordenadas de números reales, $p=(a, b, c)$. \\$\displaystyle \mathbb{R}^3 = \{(x, y, z):x, y, z \in \displaystyle \mathbb{R}\}$.

\item En $\mathbb{R}^n$ los puntos, $p \in \displaystyle \mathbb{R}^n$, se representan mediante pares ordenados de números reales, \\ $p=(x_1, x_2, ..., x_n)$.\\ $\displaystyle \mathbb{R}^n = \{(x_1, x_2, ..., 
x_n):x_1 \in \displaystyle \mathbb{R}, i=1, 2, ..., n\}$.
\vspace{3mm}

\end{itemize}

\noindent -\underline{Definición:} Geométricamente, un \textbf{vector} se define como un segmento de recta "orientado" con punto inicial en el origen de coordenadas y punto final en un punto de $\displaystyle \mathbb{R}^n$.

\vspace{5mm}

\noindent *\underline{Observación:} Si n=3, la notación habitual parala base canónica es: $\vec{i}=(1, 0, 0), \vec{j}=(0, 1, 0), \vec{k}=(0, 0, 1)$.
\vspace{3mm}

%Operaciones en Rn

\subsection{Operaciones entre elementos de $\mathbf{\displaystyle \mathbb{R}^n : \boldsymbol{\vec{x}} 
=(x_1, x_2, ..., x_n),} \mathbf{\boldsymbol{\vec{y}}=(y_1, y_2, ... y_n)}$}
\vspace{5mm}

\begin {enumerate}

\item Suma: $\vec{x}+\vec{y}=(x_1+y_1, x_2+y_2, ..., x_n+y_n)$

\item Multiplicación por un escalar, $\lambda \in \mathbb{R}: \lambda \vec{x}=(\lambda x_1, \lambda x_2, ..., \lambda x_n)$

\item Producto escalar: $<\quad ,\quad > : \displaystyle \mathbb{R}^n \times \displaystyle \mathbb{R}^n \rightarrow <\vec{x}, \vec{y}>\displaystyle \eqc{not} \vec{x} \cdot \vec{y}= \\
=  x_1y_1+x_2+y_2+...+x_ny_n = \displaystyle \sum_{i=1}^n x_i\,y_i$

\end {enumerate}

%Propiedades

\subsubsection*{Propiedades}
\vspace{3mm}

 Sean $\vec{x}, \vec{y}, \vec{z} \quad y \quad \lambda , \beta \in \displaystyle \mathbb{R}$

\begin{enumerate}

\item Linealidad: $(\lambda \, \vec{x} + \beta \, \vec{y}) \cdot \vec{z} = \lambda \, \vec{x} \, \vec{z} + \beta \, \vec{y} \, \vec{z}$

\item Conmutativa: $<\vec{x}, \vec{y}>=<\vec{y}, \vec{x}>$

\item $< \vec{x}, \vec{x} > \geq 0  \, \, \mathrm{y} < \vec{x}, \vec{x} > = 0 \iff \vec{x} = \vec{0} = (0, 0,..., 0)$

\end{enumerate}
\vspace{5mm}

%La norma euclídea

-\underline{Definición:} La \textbf{norma euclídea} de un vector $\vec{x} \in \displaystyle \mathbb{R}^n$(y se escribe $\|\vec{x}\| \mathrm{)}$ como: $\|\vec{x}\| = \sqrt{<\vec{x}, \vec{x}>} = \Big{(}\displaystyle \sum_{i = 1}^{n} x_i^{\,2}\Big{)}^{^1/_2}$.
\vspace{5mm}

\observacion Sea $\vec{x} \in \displaystyle \mathbb{R}^n$ tal que $\|\vec{x}\| = 1$, se dice que es \textbf{unitario}. Sea  $\vec{x} \in \displaystyle \mathbb{R}^n$ un vector no unitario, podemos \textbf{normalizarlo}:\\
{\large$\frac{\vec{x}}{\|\vec{x}\|} \rightarrow$} vector unitario.
\vspace{5mm}

\subsubsection*{Significado geométrico de la norma euclídea}
\vspace{3mm}

La norma $\|\vec{x}\|$ representa la \textbf{longitud} del vector $\vec{x}$. Not.: $long(\vec{x}) = |x| = d(0, x)$.
\vspace{3mm}

%Desigualdad de Cauchy-Schwarz

\subsubsection*{Desigualdad de Cauchy-Schwarz en $\mathbf{\mathbb{R}^n}$}
\vspace{3mm}

Sean $\vec{x}, \vec{y} \in \displaystyle \mathbb{R}^n$
\[
\boxed{|<\vec{x}, \vec{y}>| \leq \|\vec{x}\| \cdot \|\vec{y}\| = \bigg{|}\displaystyle \sum_{i=1}^n x_i \, 
 y_i \bigg{|} \leq \bigg{(}  \sum_{i=1}^n x_i^2 \bigg{)}^{\sfrac{1}{2}} \cdot \bigg{(} \sum_{i=1}^n 
y_i^2 \bigg{)}^{\sfrac{1}{2}}
}
\]
\vspace{5mm}

\noindent\underline{Demostración:} Tomamos $a= \vec{y} \cdot \vec{y} = \|\vec{y}\|^2 \quad \mathrm{y} 
\quad b = -\vec{x} \cdot \vec{y} \,$ y aplicamos propiedades del producto escalar al vector $a  \vec{x} + b 
\vec{y}$.

\vspace{3mm}

-Si $a=0 \Rightarrow \vec{y} = 0$  \checkmark

\vspace{1mm}

-Si $a \neq 0 \Rightarrow 0 \leq ( a\vec{x} + b \vec{y} ) (a \vec{x} + b \vec{y} ) = a^2 \vec{x}\, \vec{x} + 
2ab \vec{x} \vec{y} + b^2 \vec{y} \vec{y} = \|\vec{y}\|^4 \cdot \|\vec{x}\|^2-2 \, \|\vec{y}\|^2 \, ( 
\vec{x} \cdot \vec{y} )^2 + ( -\vec{x} \cdot \vec{y} )^2 \cdot \|\vec{y} \|^2 = \\ \indent = \|\vec{y} \|^4 
\cdot \|\vec{x}\|^2 - \|\vec{y}\|^2 \cdot ( \vec{x} \cdot \vec{y} )^2 \Rightarrow 0 \leq \|\vec{y}\|^2 \cdot 
\|\vec{x}\|^2 - ( \vec{x} \cdot \vec{y} )^2 \Rightarrow ( \vec{x} \cdot \vec{y} )^2 \leq \|\vec{x}\|^2 \cdot 
\|\vec{y}\|^2 \Rightarrow |\vec{x} \cdot \vec{y} |\leq \|\vec{x}\| \cdot \|\vec{y}\| \, \checkmark$

- $\|\vec{x}\| \cdot \|\vec{y}\| \leq < \vec{x}, \vec{y} > \leq \|\vec{x}\| \cdot \|\vec{y}\| \Rightarrow -1 \leq 
\bfrac{<\vec{x}, \vec{y}>}{\|\vec{x}\| \cdot \|\vec{y}\|} \leq 1 \Rightarrow \cos{\theta} \eqc{def} 
\bfrac{\vec{x} \cdot \vec{y}}{\|\vec{x}\| \cdot \|\vec{y}\|} \iff < \vec{x}, \vec{y} > = \\ \indent = \|\vec{x} 
\| \, \|\vec{y}\| \, \cos{\theta}$
\vspace{7mm}

- \underline{Definición:} Dos vectores $\vec{x}, \vec{y} \in \mathbb{R}^n$ son \textbf{perpendiculares/ortogonales} si y solo si $< \vec{x}, \vec{y}> = 0$. \\ Not.: $\vec{x} \perp \vec{y}$.
\vspace{3mm}

%Teorema de Pitágoras

\subsubsection*{Teorema de Pitágoras en $\mathbf{\mathbb{R}^n}$}

Sean $\vec{x}, \vec{y} \in \mathbb{R}^n$ tal que $\vec{x} \perp \vec{y}$, entonces: 
\[
\boxed{
\|\vec{x} + \vec{y}\|^2 = \|\vec{x}\|^2 + \|\vec{y}\|^2
}
\]
\vspace{3mm}

\noindent\underline{Demostración:} $\|\vec{x} + \vec{y}\|^2 = (\vec{x}+ \vec{y}) (\vec{x}+ \vec{y}) = 
\vec{x}  \vec{x} + \cancelto{0}{2 \vec{x} \vec{y}}+ \vec{y} \vec{y} = \|\vec{x}\|^2 + \|\vec{y}\|^2 \, 
\checkmark$

\vspace{3mm}

%Teorema del coseno

\subsubsection*{Teorema del coseno}

Sean $\vec{x}, \vec{y} \in \mathbb{R}^n$ vectores cualesquiera, entonces:
\[
\boxed{
\|\vec{x} - \vec{y}\|^2 = \|\vec{x}\|^2 + \|\vec{y}\|^2 - 2 \, \|\vec{x}\| \, \|\vec{y}\| \, \cos{\theta}
}
\]
\vspace{3mm}

\underline{Demostración:} $\|\vec{x} - \vec{y}\|^2 = (\vec{x} - \vec{y}) (\vec{x} - \vec{y}) = \vec{x} \vec{x} - 2 \vec{x} \vec{y} + \vec{y} \vec{y} = \vspace{1mm} \\ 
=\|\vec{x}\|^2 + \|\vec{y}\|^2 - 2 \, \|\vec{x}\| \, \|\vec{y}\| \, \cos{\theta} \, \checkmark$

\vspace{3mm}

%Propiedades de la norma euclídea

\subsubsection*{Propiedades de la norma euclídea}

Sea $\vec{x} \in \mathbb{R}^n$ y $\lambda \in \mathbb{R}$: \\ 

$(i) \|\vec{x}\| \geq 0$ y $ \|\vec{x}\| = 0 \iff \vec{x} = \vec{0}$ \vspace{3mm}

$(ii) \| \lambda \vec{x}\| = |\lambda| \cdot \|\vec{x}\|$ \vspace{3mm}

$(iii)$Desigualdad tiangular: $\|\vec{x} + \vec{y}\| \leq \|\vec{x}\| + \|\vec{y}\|$ \vspace{3mm}

$(iv)$Desigualdad triangular al revés: $\big{|} \|\vec{x}\| - \|\vec{y}\|\big{|} \leq \|\vec{x}- \vec{y}\|$
\vspace{5mm}

\underline{Demostraciones:}
\vspace{5mm}

\dindent $(i) \, \|\vec{x}\| = \sqrt{x_1^2 + x_2^2+...+ x_n^2} \geq 0$ y $ \|\vec{x}\| = \sqrt{x_1^2 + 
x_2^2 + ...+ x_n^2} = 0 \iff x_1, x_2,..., x_n = 0  \Rightarrow \\ \dindent \Rightarrow \vec{x} = \vec{0}$
\vspace{3mm}

\dindent $(ii) \, \displaystyle \|\lambda \vec{x}\| = \Big{(} \sum_{i=1}^n \lambda ^2 \, x_i^2 \Big{)} 
^{\sfrac{1}{2}} = \sqrt{\lambda^2} \, \Big{(} \sum_{i=1}^n x_i^2 \Big{)}^{\sfrac{1}{2}} = |\lambda| \, 
\|\vec{x}\|$
\vspace{3mm}

\dindent $(iii) \, \|\vec{x} + \vec{y}\|^2 = (\vec{x} + \vec{y})(\vec{x} + \vec{y}) = \vec{x} \vec{x} + 2 
\vec{x} \vec{y} + \vec{y} \vec{y} \leqc{C-S} \|\vec{x}\|^2 + 2 \, \|\vec{x}\| \, \|\vec{y}\| + \|\vec{y}\|^2 
(\|\vec{x}\| + \|\vec{y}\|)^2 \Rightarrow \\ \dindent \Rightarrow \|\vec{x} + \vec{y}\| \leq \|\vec{x}\| + 
\|\vec{y}\|$ 
\vspace{7mm}

\noindent-\underline{Definición:} Una \textbf{distancia/métrica} es una función $d:\mathbb{R}^n \times \mathbb{R}^n \rightarrow \mathbb{R}^+$ tal que cumple: \vspace{3mm}

(1) $ d(x, y) = 0 \iff x=y$ \vspace{3mm}

(2) $ d(x, y) = d(y, x)$ \vspace{3mm}

(3) $d(x, z) \leq d(x, y) + d(y, z)$
\vspace{7mm}

\noindent-\underline{Definición:} La \textbf{distancia euclídea} entre dos puntos $x = (x_1, x_2,..., x_n)$ e 
$y = (y_1, y_2,..., y_n)$ de $\mathbb{R}^n$ se define como:

\[
\boxed{
d(x, y) = \sqrt{(x_1-y_1)^2+(x_2+y_2)^2+...+(x_n+y_n)^2}=\bigg{(} \sum_{i=1}^n (x_i - y_i)^2 \bigg{)} 
^{\sfrac{1}{2}} = \|\vec{x} - \vec{y}\|
}
\]
\vspace{3mm}

(1) $\|\vec{x} - \vec{y}\| = 0 \iff \vec{x} = \vec{y}$
\vspace{3mm}

(2) $\|\vec{x} - \vec{y}\| = \|\vec{y} - \vec{x}\|$
\vspace{3mm}

(3) $\|\vec{x} - \vec{z}\| \leq \|\vec{x} - \vec{y}\| + \|\vec{y} - \vec{z}\|$
\vspace{5mm}

%CONCEPTOS MÉTRICOS EN EL ESPACIO EUCLÍDEO

\section{Conceptos métricos en el espacio euclídeo en $\mathbb{R}^{\mathbf{n}}$}
\vspace{3mm}

%BOLAS ABIERTAS Y CONJUNTOS ABIERTOS EN Rn

\subsection{Bolas abiertas y conjuntos abiertos en $\mathbb{R}^{\mathbf{n}}$}
\vspace{3mm}

-\underline{Definiciones:}
\vspace{5mm}

(1) Dado un punto $x_0 \in \mathbb{R}^n$ y un número real $r>0$, llamaremos \textbf{bola abierta} de 
centro $x_0$ y radio \\ \indent $r$ (y escribiremos $B_r (x_0) $ o $B(x_0, r)$al conjunto:
\[
\boxed{
B_r (x_0)  \eqc{not} B(x_0, r) = \{ x \in \mathbb{R}^n : \|x-x_0\|<r\}
}
\]
\vspace{3mm}

$n=1 \rightarrow x_o \in \mathbb{R} \quad -r<x-x_o<r$

$B_r(x_0) = \{ x \in \mathbb{R} : |x-x_0| <r\} = \{ x \in \mathbb{R} : x \in (x_0-r, x_0+r) \}$
\vspace{5mm}

$n=2 \rightarrow (x_0, y_0) \in \mathbb{R}^2$

$B_r (x_0, y_0) = \{ (x, y) \in \mathbb{R}^2 : \underbrace{\|(x-y) - (x_0 - y_0) \|}_{(x-x_0)^2 + (y + 
y_0)^2} < r\}$

$n=3 \rightarrow (x_0, y_0, z_0) \in \mathbb{R}^3$

$B_r (x_0, y_0, z_0) = \{ (x, y, z) \in \mathbb{R}^3 : (x-x_0)^2 + (y - y_0)^2 + (z - z_0)^2 < r$
\vspace{5mm}

(2) Decimos que $A \in \mathbb{R}^n$ es un \textbf{conjunto abierto} si $\forall x_0 \in A \, \, \exists r>0$ tal que $B_r (x_0) \subset A$.
\vspace{5mm}

(3) Un \textbf{entorno} del punto $x \in \mathbb{R}^n$ es un conjunto abierto que lo contiene.
\vspace{7mm}

%Proposiciones

\proposicion[1] Si $x_0 \in \mathbb{R}^n$ y $r>0 \Rightarrow$ la bola abierta $B_r(x_0)$ es un conjunto abierto.
\vspace{5mm}

\indent \underline{Demostración:} $\forall x \in B_r(x_0) \, \exists s \in \R , s>0$ tal que $B_s(x) \subset B_r(x_0)$.
\vspace{3mm}

\dindent Basta con tomar $s = r - \|x-x_0\|$, entonces $\forall y \in B_s(x) \quad \|y - x_0\| = \\
\dindent = \|y -x +x -x_0\| \leq \|y-x\| + \|x - x_0\| < s + \|x - x_0\| = r - \|x - x_0\| + \|x - x_0\| = r$
\vspace{5mm}

\underline{Ejemplo:} Demostrar que $A = \{ (x, y) \in \R^2 : y>0\}$ es un conjunto abierto.
\vspace{3mm}

$\forall (x_0, y_0) \in A \spac \exists r : B_r(x_0, y_0) \subset A$, es decir, $y_0 > 0$

Basta con tomar $r = y_0 > 0$.
\vspace{3mm}

$\forall (a, b) \in B_r(x_0, y_0) \Rightarrow$ ¿$(a, b) \in A$, es decir, $b>0$?
Como $ (a, b) \in B_r(x_0, y_0) \Rightarrow$

$\Rightarrow \|(a, b) - (x_0, y_0)\| < r = y_0 \Rightarrow (a - x_0)^2 + (b - y_0)^2 < y_0^2 \Rightarrow 
(b - y_0)^2 < y_0 ^2 \Rightarrow$

$\Rightarrow -y_0 < b-y_0 < y_0 \Rightarrow 0< b < 2 y_0 \Rightarrow b > 0 \Rightarrow (a, b) \in A$
\vspace{5mm}

\proposicion[2] La unión (finita o infinita) de conjuntos abiertos es un conjunto abierto.
\vspace{5mm}

\proposicion[3] La intersección finita de abiertos es un conjunto abierto.
\vspace{5mm}

\observacion La intersección finita de abiertos no es, en general, un abierto.
\vspace{3mm}

\underline{Demostración:} $\{A_1, A_2,..., A_n\}$ conjuntos abiertos y $A = \displaystyle \bigcap_{i = 1}^n A_i$.  ¿A es un conjunto abierto? 

\dindent Dado $x \in A \Rightarrow x \in A_1, x \in A_2,..., x \in A_n \Rightarrow \exists r_1, r_2,..., r_n$ tal que $ B_{r_1}(x) \subset  A_1, B_{r_2}(x) \subset$

\dindent $\subset A_2,..., B_{r_n}(x) \subset A_n$, tomando $r \leq min \{r_i\}^{n_i=1} \Rightarrow B_r(x) 
\subset \displaystyle \bigcap_{i = 1}^n A_i = A$.
\vspace{3mm}


%LIMITES DE SUCESIONES EN Rn. COMPLETITUD.

\subsection{Límites de sucesiones en $\mathbf{\R^n}$. Completitud.}
\vspace{3mm}

\definicion Una \textbf{sucesión en} $\mathbf{\R^n}$ es una colección de n sucesiones en $\R 
: \{A_k\}_{k \in \N} = \{(a_1^k, a_2^k,..., a_n^k)\}_{k \in \N}$.
\vspace{3mm}

\underline{Ejemplos:} $\{A_k\}_{k \in \N} = \left\{ \left (k, \bfrac{1}{k^2}, e^{-k}\right ) \right \}_{k \in \N}; \{A_k\}_{k \in \N} = \left \{ \left (\bfrac{3k^2}{k^2-1}, \bfrac{1}{k}, \log \bfrac{k+1}{k} \right 
)\right \}_{k \in \N}$.
\vspace{5mm}

\definicion Dada una sucesión $\{A_k\}_{k \in \N} \in \R^n$ y un punto $L \in \R^n$, 
diremos que el límite de $\{A_k\}_{k \in \N}$ es $L$ y escribimos: 
\[
\boxed{\displaystyle \lim_{k \to \infty} A_k = L \iff \forall \epsilon > 0 \spac \exists N \in \N : \|A_k - L\| < 
\epsilon \spac \forall k > N\text{, es decir, } \forall k > N \spac A_k \in B_\epsilon (l)}
\]

%LEMA

\lema Dada una sucesión en $\R^n : \{A_k\}_{k \in \N} = \{(a^k_1, a^k_2,..., a^k_n)\}_{k \in \N}$ y un 
punto $L = (l_1, l_2,..., l_n) \in \R^n$, se tiene:

\[
\boxed{\displaystyle \lim_{k \to \infty} A_k = L \iff \lim_{k \to \infty} a_i^k = l_i \spac \forall i = 1, 2,..., n}
\]
\vspace{3mm}

\underline{Demostración:}
\vspace{3mm}

$\Rightarrow$) Queremos ver que $\forall \epsilon > 0 \spac \exists N \in \N : |a_i^k - l_i| < \epsilon \spac 
\forall k > N \forall i = 1, 2,..., n$.
\vspace{3mm}

$|a_i^k - l_i| =\sqrt{(a_i^k - l_i)^2} \leq \sqrt{(a_1^k - l_1)^2 + (a_1^k - l_2)^2 + ... + (a_i^k - l_i)^2 +... 
+ (a_n - l_n)^2} =
$

$ = \|A_k - L\| \leq \epsilon \spac \checkmark$
\vspace{3mm}

$\Leftarrow$) Como $\displaystyle \lim_{k \to \infty} a_i^k = l \spac \forall i = 1, 2,..., n$, sabemos que $\forall \epsilon > 0 :$
\vspace{3mm}

$
\begin{array}{c}
\exists N_1 \in \N \text{ tal que } |a_1^k - l_1| < \bfrac{\epsilon}{\sqrt{n}} \spac \forall k > N_1 
\vspace{1mm}\\
\exists N_2 \in \N \text{ tal que } |a_2^k - l_2| < \bfrac{\epsilon}{\sqrt{n}} \spac \forall k > N_2 \\
\vdots\\
\exists N_n \in \N \text{ tal que } |a_n^k - l_n| < \bfrac{\epsilon}{\sqrt{n}} \spac \forall k > N_n\\
\end{array}
$
\vspace{3mm}

\noindent Entonces, $\forall k \geq \max \{N_1, N_2,..., N_n\}$ tenemos simultáneamente que $|a_i^k - l_k| 
< \bfrac{\epsilon}{\sqrt{n}} \spac \forall i = 1, 2,..., n$, por tanto:
\vspace{1mm}

\noindent$\|A_k - L\| = \sqrt{(a_1^k - l_1)^2 + (a_1^k - l_2)^2 + ... + (a_i^k - l_i)^2 +... + (a_n - l_n)^2} 
\leq \sqrt{n \left (\bfrac{\epsilon}{\sqrt{n}}\right )^2} = \epsilon \spac \checkmark$
\vspace{3mm}

\underline{Ejemplo:} Hallar el límite $\displaystyle \lim_{k \to \infty} \left \{\left ( \bfrac{\log{x}}{x^2}, 
\bfrac{x^2-1}{5 x^2} \right )\right \}_{k \in \N}$.
\vspace{3mm}

\[
\left.\begin{array}{lc}\displaystyle

\lim_{k \to \infty} \bfrac{\log{k}}{k^2} \eqc{L'H} \lim_{k \to \infty} \bfrac{\sfrac{1}{k}}{2 k} = \lim_{k \to 
\infty} \bfrac{1}{2 k^2} = 0 \\

\lim_{k \to \infty} \bfrac{x^2 - 1}{5 x^2} = \bfrac{1}{5}

\end{array}\right \}
\displaystyle \lim_{k \to \infty} \left \{\left ( \bfrac{\log{x}}{x^2}, \bfrac{x^2-1}{5 x^2} \right )\right \} = 
(0, 5)
\]
\vspace{5mm}

%Propiedad de completitud

\definicion La \textbf{propiedad de completitud} dice que toda sucesión de Cauchy converge en el espacio.
\vspace{5mm}

\definicion Una sucesión $\{ A_k\}_{k \in \N}$ es de \textbf{Cauchy} en $\R^n$ si $\forall \epsilon >0 \spac 
\exists N \in \N : \|A_{k_1} - A_{k_2}\| < \epsilon \spac \forall k_1, k_2 > N$.
\vspace{5mm}

\teorema El espacio euclídeo $\R^n$ es \textbf{completo}, es decir, toda sucesión de Cauchy es 
convergente.
\vspace{3mm}

\underline{Demostrción:} Pendiente como ejercicio.
\vspace{5mm}

\subsection{Más sobre la topología de $\mathbf{\R^n}$}
\vspace{5mm}

%Más sobre la topología de Rn

\definicion Sea $A$ un conjunto de $\R^n$. Diremos que un punto $x \in A$  es un \textbf{punto interior} 
de $A$ si $\exists r > 0 \text{ tal que } B_r (x) \in A$. Al conjunto de puntos interiores de $A$ se le denota $ 
\mathring{A} \eqc{not} int(A)$.
\vspace{3mm}

\underline{Ejemplo 1:} Dado $A = \{(x, y) \in \R^n : |x|<1, |y| \leq 1 \}$, hallar $\mathring{A}$.
\vspace{3mm}

\dindent 1) $\forall (x, y) \N A : |y| = 1 \Rightarrow (x, \pm 1) (\text{ con } |x| < 1) \spac (x, 1) \in A, \forall 
r > 0 \rightarrow (x, 1 + \sfrac{r}{2}) \inc{?} A \spac |1 + \sfrac{r}{2}| =$

\dindent $= 1 + \sfrac{r}{2} > 1 \Rightarrow B_r(x, 1) \not\subset A$
\vspace{3mm}

\dindent 2) $\forall (x, y) \subset A \spac |x|, |y| < 1$:

 \[
\dindent *\left. \begin{array}{rccc}

r \leq 1- |x| & \text{donde} & |y| \leq |x|\\
r \leq 1- |y| & \text{donde} & |x| \leq |y|

\end{array}\right \}\spac\parbox{10cm}{en cualquier caso $r = 1 - \max{\{|x|, |y|\}} = \min{\{1 - |x|, 
 1 - |y|\}}$}
\]
\vspace{3mm}

Tenemos que ver que $B_r (x, y) \subset A$, es decir, $\forall (a, b) \in B_r(x, y)$ se tiene que $|a| < 1 
\text{ y } |b| \leq 1 \Rightarrow$

$\Rightarrow (a, b) \in A$
\vspace{3mm}

$|a| = |a - x + x| \leqc{D.T.} \underbrace{|a-x|}_{\sqrt{(a - x)^2}} + |x| \leq \underbrace{\|(a, b) - 
(x, y)\|}_{\sqrt{(a -x)^2 + (b - y)^2}} + |x| < r + |x| \leqc{*} 1 - |x| +|x| =1$
\vspace{3mm}

$|b| = |b - y + y| \leqc{D.T.} |b - y| +|y| < r + |y| \leq 1 - |y| + |y| = 1$
\vspace{3mm}

$\forall (a, b) \in B_r(x, y)$ se tiene que $(a, b) \in A \, \checkmark$
\vspace{5mm}

%Propiedades

\noindent$\bullet$ \underline{Propiedades:}
\vspace{3mm}

(1) $\mathring{A}$ es un conjunto abierto.
\vspace{3mm}

(2) $\mathring{A} \subset A$.
\vspace{3mm}

(3) Si $A$ es un conjunto abierto, entonces $\mathring{A} = A$.
\vspace{5mm}

%Conjunto cerrado

\definicion Diremos que un conjunto $C \subset \R^n$ es un \textbf{conjunto cerrado} si su complementario 
$C^c \subset \R^n$, es un conjunto abierto.
\vspace{3mm}

\underline{Ejemplo 2:} Demostrar que la bola $C = \{ (x, y) : \underbrace{x^2 + y^2 \leq R^2}_{\|(x, y
)\|} \leq R\}$ es un conjunto cerrado en $\R^2$.
\vspace{3mm}

$C$  cerrado en $\R^2 \text{ si } C^c = \{(x, y) \in \R^2 : \underbrace{x^2 + y^2}_{\|(x, y)\|} > R\}$ es
 un conjunto abierto en $\R^2$, es decir, 

$\forall (x_0, y_0) \in C^c \spac \exists r > 0 : B_r (x_0, y_0) \subset C^c$.
\vspace{3mm}

Tomamos $r = \| (x_0, y_0) \| - R \Rightarrow \forall (a, b) \in B_r (x_0, y_0) \text{, ¿} a, b \in C^c 
\text{, es decir, } \| (a, b) \| >R$?
\vspace{2mm}

$\| (x_0, y_0) \| = \| (x_0, y_0) - (a, b) + (a, b) \| \leqc{D.T.} \underbrace{\| (x_0, y_0) - (a, b) \|}_{< r =
 \| (x_0, y_0) \| - R} + \| (a, b) \| < \| (x_0, y_0) \| - R + \| (a, b) \| \Rightarrow$

$\Rightarrow \| (x_0, y_0) \| < \| (x_0, y_0) \| - R + \| (a, b) \| \Rightarrow R < \| (a, b) \| \, \checkmark$
\vspace{7mm}

\proposicion[1]
\vspace{3mm}

(i) La unión finita de conjuntos cerrados es un conjunto cerrado.
\vspace{3mm}

(ii) La intersección (infinita o finita) de conjuntos cerrados, es un conjunto cerrado.
\vspace{5mm}

\observacion La unión infinita de conjuntos cerrados no es, en general, un conjunto cerrado.
\vspace{5mm}

-Bolas \textbf{abiertas} en $\R^2: \spac B_R (x_0, y_0) = \{ (x, y) \in \R^2 : \underbrace{(x-x_0)^2 + (y -
 y_0)^2 < R^2}_{\| (x, y) - (x_0, y_0)\| <R}\}$
\vspace{3mm}

-Bolas \textbf{cerradas} en $\R^2:  \spac \overline{B_R (x_0, y_0)} = \{(x , y) \in \R^2 : \underbrace{(x -
 x_0)^2 + (y - y_0)^2 \leq R^2}_{\|(x, y) - (x_0, y_0)\| \leq R}\}$
\vspace{3mm}

\observacion[1] No todos los conjuntos de $\R^n$ son necesariamente abiertos o cerrados.
\vspace{5mm}

\observacion[2] El vacío y $\R^n$ (por convención) son abiertos y cerrados al
 mismo tiempo.
\vspace{5mm}

\proposicion[2 (Caracterización de los conjuntos cerrados por sucesiones)]
\vspace{3mm}

\[
\boxed{C \subset \R^n \text{ es un cerrado } \iff \forall {A_k}_{k \in \N} \subset C \text{ con }
 \displaystyle \lim_{k \to \infty}^{} A_k = L \text{ se tiene que } \\ L \in C}
\]
\vspace{3mm}

\underline{Demostrción:} Argumentamos (en ambas implicaciones) por reducción al absurdo:
\vspace{3mm}

$\Rightarrow$ ) Supongamos que $L \not\in C \Rightarrow L \in C^c \text{ abierto } \Rightarrow \exists 
r > 0 \text{ tal que } B_r (L) \subset C^c \text{. Como} \displaystyle \lim_{k \to \infty} A_k = L,$

$\forall \epsilon > 0  \spac \exists N \in \N \text{ tal que }|A_k - L| > \epsilon \spac \forall k > N \text{, es 
decir, }A_k \in B_\epsilon (L) \spac \forall k>N \text{. Si tomamos }$

$\epsilon = r : A_k \in B_r (L)$
\vspace{3mm}

\underline{Contradicción} $A_k \in C \text{ y } B_r (L) \subset C^c$
\vspace{3mm}

$\Leftarrow$ ) Supongamos que $C$ no es un cerrado $\Rightarrow C^c$ no es un abierto $\Rightarrow \exists x_0 \in C^c \text{ tal que } \forall r>0$

$B_r (x) \not\subset C^c \text{, es decir, }
B_r (x_0) \cap C \neq \emptyset \spac  \forall{ r > 0 }\text{. Tomamos } r =  \bfrac{1}{k}
 \Rightarrow B_{\sfrac{1}{k}} (x_0) \cap C \neq \emptyset \Rightarrow$

$\Rightarrow \exists \{A_k\}_{k \in \N} \subset B_{\sfrac{1}{k}} \cap C, \{A_k\}_{k \in \N} \rightarrow x_0 
\in C$
\vspace{3mm}

\underline{Contradicción} $x_0 \in C^c$.
\vspace{5mm}

%Punto de adherencia

\definicion Sea $C$ un conjunto cualquiera en $\R^n$ y $x \in \R^n$, diremos que 
$x$ es un \textbf{punto de adherencia o clausura} de $C$ si $\forall{r>0} \spac B_r (x) \cap C \neq \emptyset$ y el conjunto de puntos que cumplen esta condición se denota:

\[
\boxed{\overline{C} = \{x \in \R^n : \forall{r>0} \spac B_r (x 
\cap C \neq \emptyset\}}
\]
\vspace{3mm}

\underline{Propiedades:} 
\vspace{3mm}

\dindent (1) $\overline{C}$ es un conjunto cerrado.
\vspace{3mm}

\dindent (2) $C \subset \overline{C}$.
\vspace{3mm}

\dindent (3) Si $C$ es cerrado $C = \overline{C}$.
\vspace{5mm}

\underline{Ejemplo 3:} Sea $C = \left \{ \left ( \bfrac{1}{k} , \bfrac{1}{k} \right ) \right \}_{k \in \N}
$probar que $C \in \overline{C}$.
\vspace{3mm}

\dindent $\displaystyle \lim_{k \to 0} \left (\bfrac{1}{k} , \bfrac{1}{k} \right ) = (0, 0) \Rightarrow 
\forall{\epsilon > 0} \exists N \text{ tal que  } \| \left (\bfrac{1}{k} , \bfrac{1}{k} \right ) - (0, 0) \| < \epsilon 
\forall k > N.$
\vspace{3mm}

$\text{ Basta tomar } r = \epsilon \Rightarrow \left (\bfrac{1}{k} , \bfrac{1}{k} \right ) \in B_\epsilon (0, 0) 
\forall{k \geq N} \Rightarrow C \cap B_r (0, 0) \neq \emptyset$
\vspace{7mm}

%DEFINICIONES

\definicion Sea $C$ un conjunto cualquiera de $\R^n$ y $x \in \R^n$. Diremos que $x$ es un 
\textbf{punto de acumulación} de $C$ si $\forall{r > 0} \spac B_r(x) \setminus \{x\} \cap C \neq 
\emptyset $. A los puntos de acumulación se los denota por $C'$.
\vspace{5mm}

\underline{Propiedades:}
\vspace{3mm}

\indent \indent (1) $C' \subset \overline{C}$
\vspace{3mm}

\indent \indent (2) En general $\overline{C} \not\subset C'$
\vspace{5mm}

\definicion Los puntos de adherencia que no son de acumulación ($\overline{C} \setminus C'$) se llaman 
\textbf{puntos aislados}, es decir, $x \in R^n$ es un punto aislado de un conjunto$ C \subset \R^n 
\text{ si } \exists{r > 0} \text{ tal que } B_r (x) \cap C = \{x\}$.
\vspace{7mm}

\definicion Sea $C \subset \R^n$, diremos que un punto $x \in \R^n$ es un \textbf{punto frontera} de $C$ 
si $\forall{r > 0} \text{ se tiene que } \\ B_r (x) \cap C \neq \emptyset \text{ y } x \not\in \mathring{C}$.A 
este conjunto de puntos se le denota por $\partial C$.
\vspace{7mm}

\definicion Un conjunto cualquiera $C \subset \R^n$ está \textbf{acotado} si $\exists{M \in \R}, M > 0 
\text{ tal que } \forall{x \in C} \text{ se tiene que } \\ \|x\| < M \text{, es decir, } C \subset B_M (0,..., 0)$.
\vspace{7mm}

\definicion Un \textbf{recubrimiento numerable por abiertos} de un conjunto $S 
\subset \R^n$ es una colección numerable de abiertos $\{A_i\} \text{ tal que } S \subset \displaystyle 
\bigcup_{i=1}^\infty A_i$.
\vspace{5mm}

\definicion Un conjunto es \textbf{compacto} si para todo recubrimiento por abiertos 
del conjunto se puede extraer un subrecubrimiento finito.
\vspace{5mm}

%Teorema de Heine-Borel

\teorema[de Heine-Borel]$C \subset \R^n$ es compacto $\iff C$ es cerrado 
y acotado en $\R^n$.
\vspace{5mm}

\underline{Ejemplo 1:} $B_r (x_0, y_0)$ no compactos (n son cerrados).
\vspace{5mm}

\underline{Ejemplo 2:} $\overline{B_r (x_0, y_0)}$ si son compactos.
\vspace{5mm}

\underline{Ejemplo 3:} $S = \left \{ \left ( \bfrac{(-1)^k}{k}, \bfrac{1}{k^2} \right ) \right  \}_{k \in \N}$
\vspace{3mm}

\dindent Por la proposición 2, como $\displaystyle \lim_{k \to \infty} \left ( \bfrac{(-1)^k}{k},
 \bfrac{1}{k^2} \right ) = (0, 0) \text{ y } (0, 0) \not\in S$,  no es  cerrado y por el \dindent teorema de 
Heine-Borel no es compacto.
\vspace{5mm}

\underline{Ejemplo 4:} $S = \{(k, k^2+1)\}_{k \in \N}$ no está acotado, por tanto por el teorema de 
Heine-Borel \\ \indent tampoco es compacto.
\vspace{7mm}

%Teorema de Bolzano-Weierstrass

\teorema[de Bolzano-Weierstrass] Si $C$ es un conjunto de infinitos elementos y acotado $\Rightarrow C' 
\neq \emptyset$. Es decir, toda sucesión $\{A_k\}_{k \in \N}$ definida en un compacto $C$ posee una 
subsucesión convergente en $C$.
\vspace{5mm}

%FUNCIONES DE VARIAS VARIABLES Y SUPERFICIES DE NIVEL

\subsection{Funciones de varias variables y superficies de nivel}
\vspace{5mm}

\definicion Llamamos \textbf{función de varias variables} a la función:

\[
\boxed{
\begin{array}{c}

f : D \subset \R^n \to \R^m \\
(x_1, x_2, x_n) \to (f_1(x_1,..., x_n),..., f_n(x_1,...x_n))\\

\end{array}
}
\]
\vspace{3mm}

Si $m > 1$, entonces $f (x_1, x_2,..., x_n) \in \R$ y se llama \textbf{función escalar}.
\vspace{3mm}

Si $m > 1$, entonces se llama \textbf{función vectorial}.
\vspace{7mm}

\definicion Sea $f : D \subset \R^n \to \R$, llamaremos \textbf{gráfica} de $f$ al subconjunto de $\R^
{n+1} : \\ (x_1, x_2,..., x_n, f (x_1, x_2,..., x_n)) \in \R^{n+1}$.
\vspace{5mm}

\definicion Sea $f : D \subset \R^n \to \R \text{ y  } C \in \R$, el \textbf{conjunto de nivel} del valor $C$ son 
los puntos : \\ $\{x \in D : f (x) = C\}$.
\vspace{3mm}

$n = 2$ : curvas de nivel.
\vspace{3mm}

$n = 3$ : superficies de nivel.
\vspace{5mm}

\underline{Ejemplo 1:} Usando curvas de nivel. describir la gráfica $f (x, y) = x^2 + y^2$.
\vspace{3mm}

\dindent Curvas de nivel: $\{ (x, y) \in \R^2 : x^2 + y^2 = C, C \geq 0\}$
\vspace{3mm}

\dindent$C = 0 \to x^2 + y^2 = 0 \Rightarrow (x, y) = 0$
\vspace{3mm}

\dindent$C = 1 \to x^2 + y^2 = 1^2 \Rightarrow$ Circunferencia de radio 1
\vspace{3mm}

\dindent$C = 4 \to x^2 + y^2 = 2^2 \Rightarrow$ Circunferencia de radio 2
\vspace{3mm}

\dindent$C = 9 \to x^2 + y^2 = 3^2 \Rightarrow$ Circunferencia de radio 3
\vspace{5mm}

\underline{Ejemplo 2:} Describir las superficies de nivel de $f (x, y, z) = x^2 + y^2 + z^2$.
\vspace{3mm}

\dindent Superficies de nivel: $\{ (x, y, z) \in \R^3 : x^2 + y^2 + z^2 = C, C \geq 0\} =$ esferas de radio $ 
\sqrt{C}$.
\vspace{7mm}

%TEMA 2. CÁLCULO DIFERENCIAS EN VARIAS VARIABLES

\part*{Tema 2. Cálculo diferencial en varias variables}
\vspace{5mm}

\setcounter{section}{0}

%Límites y continuidad

\section{Límites y continuidad}
\vspace{5mm}

%Límites

\subsection{Límites}
\vspace{5mm}

\definicion Sea $A  \subset \R^n$ un conjunto abierto, $x_0 \in A' \text{ y } f : A \subset \R^n \to \R$ una
función escalar. Dado $L \in \R$ diremos que el \textbf{límite} de $f (x)$ es $L$ cuando $x$ tiende a $x_0$:
\[
\boxed{
\displaystyle \lim_ {x \to x_0}^{} f(x) = L \iff \forall{\epsilon > 0} \spac \exists{\delta > 0} : 
 \|x - x_0\| < \delta \Rightarrow |f (x) - L| < \epsilon \spac \forall{x \in A}
}
\]
\vspace{3mm}

O lo que es lo mismo:

\[
\boxed{
\displaystyle \lim_{x \to x_0}^{} f (x) =L \iff \forall{x \in B_\delta (x_0)} \Rightarrow f (x) \in B_\epsilon(L)
}
\]
\vspace{3mm}

\underline{Ejemplo 1:} Demuestra, usando la definición, que $\displaystyle \lim_{(x, y) \to (0, 0)} 2x^2 + 
y^2 = 0$.
\vspace{3mm}

\dindent{$\forall{\epsilon > 0} \spac \exists \delta > 0 : \| (x, y) - (0, 0) \| < \delta \Rightarrow 
| (2x^2 + y^2) - 0 | < \epsilon$}
\vspace{3mm}

\dindent$\underbrace{\sqrt{x^2 + y^2} < \delta}_{x^2 + y^2 < \delta^2} \Rightarrow | 2x^2 + 
y^2 | < 2 (x^2 + y^2) < 2\delta^2 < \epsilon$, por tanto bastará con tomar $\delta <
\sqrt{\bfrac{\epsilon}{2}}$.
\vspace{5mm}

\underline{Ejemplo 2:} Demuestra, usando la definición, que $\displaystyle \lim_{ (x, y) \to (0, 0) } 
\bfrac{5x^2y}{x^2 + y^2} =0$.
\vspace{3mm}

\dindent $\forall{\epsilon > 0} \spac \exists{\delta > 0} : \underbrace{\| (x, y) - (0, 0) \| < \delta}_
{\sqrt{x^2 + y^2} < \delta} \Rightarrow \left | \bfrac{5x^2y}{x^2 + y^2} \right |< \epsilon$
\vspace{3mm}

\dindent$\left |\bfrac{5x^2 y}{x^2 + y^2} \right | \leq \left | \bfrac{5x^2 y}{x^2} \right | =  
\underbrace{|5 y|}_{5 \sqrt{y^2}} \leq 5 \underbrace {\|(x, y) - (0, 0)\|}_{\sqrt{x^2 + y^2}} < 5 \delta < 
\epsilon$, por tanto bastará con tomar $\delta < \bfrac{\epsilon}{5}$.
\vspace{7mm}

\teorema El límite de una función $f (x)$, si existe, es único.
\vspace{5mm}

\underline{Demostración:} $\displaystyle \lim_{x \to x_0} f (x) = L_1 \neq L_2 = \lim_{x \to x_0} f (x)$
\vspace{3mm}

\dindent$2 \epsilon < |L_1 - L_2| = |L_1 - f(x) + f(x) - L_2| \leq \underbrace{|L_1 - f(x)|}_{< \epsilon} + \underbrace{|f (x) - L_2|}_{< \epsilon} < 2 \epsilon$

\dindent\underline{Contradicción:} $2 \epsilon < 2 \epsilon$
\vspace{7mm}

\underline{Propiedades:} Sean $f, g : A \text{(abierto)} \subset \R^n \to \R$, $\displaystyle \lim_{x \to x_0} 
f (x) = L_1 \text{ y } \lim_{x \to x_0} g (x) = L_2$:
\vspace{3mm}

\dindent (1) $\displaystyle \lim_{x \to x_0} (f (x) \pm g (x)) = L_1 + L_2$
\vspace{3mm}

\dindent (2) $\displaystyle \lim_{x \to x_0} (f (x) \cdot g (x)) = L_1 \cdot L_2$
\vspace{3mm}

\dindent (3) $\displaystyle \lim_{x \to x_0} \bfrac{f (x)}{g (x)} = \bfrac{L_1}{L_2} \text{ si } L_2 \neq 0 
\text{ y } g (x) \neq 0$ en un entorno de $x_0$.
\vspace{3mm}

\dindent (4) $\displaystyle \lim_{x \to x_0} \lambda f (x) = \lambda L_1$
\vspace{3mm}

\dindent (5) $\displaystyle \lim_{x \to x_0} [f (x)]^{g (x)} = L_1^{L_2}$ si tiene sentido.
\vspace{7mm}

\observacion Si $\exists \displaystyle \lim_{x \to x_0} f (x) = L \Rightarrow f (x)$ está acotada en un entorno 
de $x_0$.
\vspace{5mm}

\underline{Demostracion:} $| f (x) | - L \leq |f (x) - L|< \epsilon \Rightarrow |f (x)| \leq L + \epsilon$
\vspace{7mm}

\proposicion[1] Sea $f : A \text{ (abierto) }  \subset \R^n \rightarrow \R^m \text{ una función 
vectorial , } \vspace{1mm} \\  x_0 \in A' \text{ y } L \in L \in \R^m$, entonces:
\[
\boxed{
\displaystyle \lim_{x \to x_0}^{} f (x_1, x_2,...,x_n) = L \iff \lim_{x \to x_0} f_i (x_1, x_2,..., x_n) = l_i
}
\]
\vspace{3mm}
\[
\left \{\begin{array}{lc}
f (x_1, x_2,...,x_n) = (f_1 (x_1, x_2,..., x_n), f_2 (x_1, x_2,..., x_n),..., f_m (x_1, x_2,..., x_n)) \\
L = (l_1, l_2,..., l_m)
\end{array}\right.
\]
\vspace{10mm}

%Caracterización de límites de funciones por sucesiones

{\color{red}\proposicion[2] "Caracterización de límites de funciones por  sucesiones"}
\vspace{3mm}

Sea $f : A \text{(abierto)} \subset \R^n \to \R, x_0 \in A' \text{ y } L \in \R$, entonces:

\[
\boxed{
\displaystyle \lim_{x \to x_0}^{} f (x) = L \iff \forall \{x_k\}_{k \in \N} \in A \lim_{k \to \infty} x_k = x_0 \text{ y } x_k \neq x_0 \Rightarrow \lim_{k \to \infty} f (x_k) = L
}
\]
\vspace{3mm}

\underline{Demostración:}
\vspace{3mm}

\dindent $\Rightarrow$) $\forall \{x_k\} \text{ tal que } \displaystyle \lim_{k \to \infty} x_k = x_0$, ¿se tiene 
que $\displaystyle \lim_{k \to \infty} f (x_k) = L$?
\vspace{3mm}

\dindent Por convergencia de sucesiones tenemos 	que $\forall{\epsilon_1 > 0} \spac \exists{N_1 \in \N} : 
\|x_k - x_0\| < \epsilon_1 \spac \forall{k \geq N_1} \\ \dindent\text{, tomando } \epsilon_1 = \delta 
\Rightarrow |f (x_k) - L| < \epsilon$.
\vspace{3mm}

\dindent $\Leftarrow$) Argumentamos por reducción a lo absurdo. 
\vspace{3mm}

\dindent Suponemos que $\displaystyle \lim_{x \to x_0} f (x) \neq L \text{, es decir, } \exists{\epsilon_2 > 0} \text{ tal que } \forall{\delta > 0} \text{  se tiene que } \|x - x_0\| < \delta \Rightarrow \\ \dindent 
\Rightarrow|f (x) - L| < \epsilon_2 \text{( es decir, } f (b_\delta) \cap B_{\epsilon_2} (L) \neq \emptyset)$.
\vspace{3mm}

\dindent Se cumple $\forall \delta$, en particular para $\delta = \left \{ \bfrac{1}{k} \right \} \Rightarrow 
\text{tomamos una sucesión } \{x_k\} \in b_{\sfrac{1}{k}} (x_0) \text{ tal que } \\ \dindent \displaystyle 
 \lim_{k \to \infty} x_k = x_0 \Rightarrow |f (x_k) - L| < \epsilon$
\vspace{3mm}

\dindent \underline{Contradicción:} $(II \Rightarrow I)$.
\vspace{5mm}

\observacion La caracterización de límites por sucesiones será muy útil para demostrar la NO existencia de 
límites, encontrando dos sucesiones con límites diferentes.
\vspace{7mm}

%Continuidad

\subsection{Continuidad}
\vspace{5mm}

\definicion Una función $f : A \text{ (abierto) } \subset \R^n \to \R \text{ es \textbf{continua} en } x_0 \in 
A$ si:

\[
\boxed{
\forall{\epsilon > 0} \spac \exists{\delta > 0} : \|x - x_0\| < \delta \Rightarrow |f (x) - f (x_0)| 
< \epsilon
}
\]
\vspace{3mm}

Es decir:
\vspace{3mm}

(1) $\exists f (x_0)$
\vspace{3mm}

(2) $\exists \displaystyle \lim_{x \to x_0} f (x)$
\vspace{2mm}

(3) $f (x_0) = \displaystyle \lim_{x \to x_0} f (x)$
\vspace{5mm}

\observacion Una función vectorial $f : A \text{ (abierto) } \subset \R^n \to \R^m \text{ es continua } \iff 
f_i \text{ es continua } \forall{i = 1, 2,..., m}$.
\vspace{5mm}

\underline{Propiedades:} Sean $f \text{ y } g$ funciones continuas en $x_0$, entonces:
\vspace{3mm}

\indent \indent (1) $(f \pm g) (x) \text{ es continua en } x_0$.
\vspace{3mm}

\indent \indent (2) $(\lambda f) (x) \text{ es continua en } x_0 \spac \forall{\lambda \in \R}$.
\vspace{3mm}

\indent \indent (3) $(f \cdot g) (x) \text{ es continua en } x_0$.
\vspace{3mm}

\indent \indent (4) $\left ( \bfrac{f}{g} \right ) (x) \text{ es continua en } x_0 \text{ si } g (x_0) \neq 0$.
\vspace{3mm}

\indent \indent (5) $[f (x)]^{g (x)} \text{ es continua en } x_0$ si tiene sentido.
\vspace{7mm}

%Caracterización de funciones continuas por sucesiones

\proposicion[1] "Caracterización de las funciones continuas por sucesiones"
\vspace{3mm}

Sea $x_0 \in A \text{ y } f : A \subset \R^n \to \R$:

\[
\boxed{
f \text{ continua en } x_0 \iff \forall{\{x_k\}_{k \in \N}} : \displaystyle \lim_{k \to  \infty} x_k = x_0 \text{ y }
 x_k \neq x_0 \Rightarrow \lim_{k \to \infty} f (x_k) = f (x_0)
}
\]
\vspace{3mm}

\underline{Ejemplo 1:} ¿Es continua en $\R^2$ la siguiente función $f$?:

\[
f (x, y) = \left \{
\begin{array}{ccc}

\bfrac{5x^2y}{x^2 + y^2} & \text{ si } & (x, y) \neq (0, 0) \\
0 & \text{ si } & (x, y) = (0, 0)\\

\end{array}\right.
\]
\vspace{3mm}

(1) $f (0, 0) = 0$
\vspace{3mm}

(2) $\displaystyle \lim_{(x, y) \to (0, 0)} \bfrac{5x^2y}{x^2 + y^2} = 0$
\vspace{3mm}

(3) $f (0, 0) = \displaystyle \lim_{(x, y) \to (0, 0)} f (x, y)$
\vspace{3mm}

$f (x, y) \text{ es continua en todo } \R^2$.
\vspace{7mm}

\teorema Sean $f : \R^n \to \R^m \text{ y } g : \R^m \to \R^p$:

\[
\boxed{
f (x) \text{ continua en } x_0 \in \R^n \text{ y } g (y) \text{ continua en } f (x_0) \in R^m \Rightarrow 
(g \circ f) (x) \text{ continua en } x_0
}
\]
\vspace{3mm}

%Conjuntos abiertos, cerrados y funciones continuas
\vspace{5mm}

\subsection{Conjuntos abiertos, cerrados y funciones continuas}
\vspace{3mm}

\definicion Sea $f : \R^n \to \R^m, \, X \subset \R^n \text{ e } Y \subset \R^m$, se define la 
\textbf{imagen inversa} o \textbf{preimagen} del conunto $Y$ mediante la función $f$ como:
\[
\boxed{
f^{-1} (Y) = \{x \in \R^n : f (x) \subset Y\}
}
\]
\vspace{3mm}

\underline{Propiedades:}
\vspace{3mm}

\dindent (1) Si $X = f^{-1} (Y) \Rightarrow f (X) \subset Y \Rightarrow f (f^{-1} (Y)) \subset Y$
\vspace{3mm}

\tindent \underline{Ejemplo:} $f = 
\left \{ \begin{array}{ccc}

-1 & \text{si} & x < 0\\
1 & \text{si} & x>0\\

\end{array}\right.$
\vspace{2mm}

\tindent \indent $f^{-1} \underbrace{([0, 2])}_{Y} : \underbrace{[0, \infty)}_{X} \Rightarrow f ([0, +\infty)) 
= {1} \subset [0, 2]$
\vspace{3mm}

\dindent (2) Si $Y = f (X) \Rightarrow X \subset f^{-1} (Y) \Rightarrow X \subset f^{-1} (f (X))$
\vspace{3mm}

\tindent \underline{Ejemplo:} $f (x) = x^2$
\vspace{3mm}

\tindent \indent $f (\{2\}) = 4 \Rightarrow \{2\} \subset f^{-1} (\{4\}) = \{2, 2\}$
\vspace{5mm}

\underline{Ejemplo 1:} $
f = 
\left \{ \begin{array}{ccc}

-1 & \text{si} & x < 0\\
1 & \text{si} & x>0\\

\end{array}\right.
$
\vspace{3mm}

\[
\begin{array}{lccrc}

f^{-1} (\{1\}) = [0, +\infty) & \quad & f^{-1} (\{-1\}) = (- \infty, 0) \vspace{3mm}\\
f^{-1} ([-1, 1]) = \R & \quad & f^{-1} ([0, 2]) = [0, +\infty)\vspace{3mm}\\
f^{-1} ([-2, 0]) = (-\infty, 0) & \quad &\\

\end{array}
\]
\vspace{3mm}

\underline{Ejemplo 2:} Sea $f (x, y) = 2x^2 + y^2 - 4 \text{ ¿}f^{-1} (\{3\}), f^{-1} ([-\infty, 0])\text{?}$
\vspace{3mm}

\dindent $f^{-1} (\{3\}) = \{(x, y) \in \R^2 : 2x^2+ y^2 = 1\} \rightarrow$ Elipse de altura $z =1$
\vspace{3mm}

\dindent $f^{-1} ((-\infty, 0]) = \{(x, y) \in \R^2 : 2x^2+ y^2 \leq 4\} \rightarrow$ Elipses de alturas desde 
0 hasta -4.
\vspace{3mm}

\dindent Esto tiene sentido dado que la función forma un paraboloide elíptico.
\vspace{7mm}

\teorema[1] Dada $f : \R^n \to \R^m$:

\[
\boxed{
f \text{ es continua} \iff \forall \, V \subset \R^m \text{ abierto } : f^{-1} (V) \subset \R^n \text{ es 
abierto }
}
\]
\vspace{3mm}

\underline{Demostrción:} $\Rightarrow$) Hipótesis $
\left \{\begin{array}{lc}

\text{(I)} \, \forall{\epsilon > 0} \spac \exists{\delta > 0} : f (B_\delta (x)) \subset B_\epsilon (f (x))\\
\text{(II)} \, \forall y \in V \subset \R^m \spac \exists \delta > 0 \spac B_\delta (y) \subset V \subset \R^m\\

\end{array} \right.
$
\vspace{5mm}

\dindent ¿$f^{-1} (V) \text{ abierto en } \R^n \text{, es decir, } \forall{x \in f^{-1}} \spac \exists{r > 0} : 
B_r (x) \subset f^{-1} (V)$?

\dindent $\forall{x \in f^{-1} (V)} \spac \exists{y \in V} : y = f (x) \subset V \xRightarrow{\text{I}} \epsilon = 
\delta \Rightarrow B_\epsilon ( f (x)) \subset V$.

\dindent Por II $f (B_\delta (x)) \subset B_\epsilon ( f(x)) \in V$ y tomamos preimágenes $B_\delta (x) 
\gsc{\subset}{Prop. (2)} f^{-1} (f (B_\delta (x))) \subset \\ \dindent \subset f^{-1} (B_\epsilon ( f (x))) 
\subset f^{-1} (V) $. 
\vspace{3mm}

\dindent Basta con $r = \delta$. \checkmark
\vspace{5mm}

\dindent $\Leftarrow$) Hipótesis: Sea $f : \R^n \to \R^m$ se tiene que $V \in \R^m \text{ abierto } 
\Rightarrow f^{-1} (V) \text{ abierto en } \R^n$.
\vspace{3mm}

\dindent ¿$f \text{ es continua, es decir, } \forall{ \epsilon > 0} \spac \exists{ \delta > 0} : f (B_\epsilon (x)) 
\subset B_\epsilon (f (x))$? 
\vspace{3mm}

\dindent $\forall{x \in f^{-1} (V)} \subset \R^n \exists y \in V \subset \R^m : y f (x) \text{ y } B_\epsilon 
( f (x)) \text{ abierto en } \R^m \xRightarrow{hip.}f^{-1} (\underbrace{B_\epsilon (f (x))}_{V}) 
\text{ abierto en }\R^n$.


\dindent $\exists r : B_r (x) \subset f^{-1} (B_\epsilon (f (x))) \text{ tenemos } r = \delta : B_\delta (x) 
\subset f^{-1} (B_\epsilon ( f (x))) \Rightarrow \\ \dindent \Rightarrow f (B_\delta (x)) \subset f (f^{-1} 
(B_\epsilon ( f (x)))) \gsc{\subset}{Prop.(1)} B_\epsilon ( f (x))$. \checkmark
\vspace{7mm}

\underline{Ejemplo 1:} $A = \{(x , y, z) \in \R^3 : x^3 - 4y^2 + z^2 = 5\} = \{(x, y, z) \in \R^3 : 
f (x, y, z) = 5\} = \\ \indent = f^{-1} (\underbrace{\{5\}}_{\text{cerrado}}) \xRightarrow{T.2} A$ es un 
cerrado.
\vspace{5mm}

\underline{Ejemplo 2:} $B = \{(x, y) \in \R^2 : x^2 - y^2 < 1 + \cos{y}\} = \{(x, y) \in \R^2 : x^2 - y^2 - 
\cos{y} < 1\} = \\ \indent = f^{-1} ((-\infty, 1)) \Rightarrow B \text{ es un abierto en } \R^2$.
\vspace{5mm}

\underline{Ejemplo 3:} $f (x) = 7, f : \R \to \R$
\vspace{3mm}

$f (\underbrace{(-\infty, +\infty)}_{\text{abierto}}) = \{7\} \to$ cerrado.
\vspace{5mm}

\underline{Ejemplo 4:} Sea $f : [a, b] \to \R \text{ una función continua en } [a, b] \Rightarrow G (f) = 
 \{(x, f (x)) \in \R^2 : x \in [a, b]\}$ es \\ \indent un cerrado.
\vspace{3mm}

\dindent Usamos la caracterización de cerrados por sucesiones.
\vspace{3mm}

\dindent Sea $\{A_k\} = \{(x_k, y_k)\}_{k \in \N} \text{ tal que } y_k = f (x_k) \text{ y } x_k \in [a, b] 
\text{ (¡cerrado!)} :$
\vspace{2mm}

\dindent $\displaystyle \lim_{k \to \infty} \{A_k\} = (x_0, y_0) \inc{\text{?}} G (f)$.
\vspace{3mm}

\dindent Como $\displaystyle \lim_{k \to \infty} \{A_k\} = (x_0, y_0) \Rightarrow \| (x_k, y_k) - (x_0, y_0)\| < 
\epsilon \Rightarrow (x_k - x_0) + (y_k - y_0) \rightarrow 0 \Rightarrow \\ \dindent \Rightarrow
\left\{\begin{array}{lc}

\displaystyle \lim_{k \to \infty} x_k = x_0 \in [a, b]\\
\displaystyle \lim_{k \to \infty} f (x_k) = y_0 \xRightarrow{f \text{ cont}} f (x_0) = y_0\\

\end{array}\right.
$
\vspace{3mm}

\dindent $\displaystyle \lim_{k \to \infty} \{(x_k, \underbrace{f (x_k)}_{\in G (f)})\} = \{(x_0, f (x_0)) : x_0 \in [a, b]\} \spac \exists G (f) \rightarrow$ cerrado. \checkmark
\vspace{7mm}

\teorema[2] Dada $f : \R^n \to \R^m$:

\[
\boxed{
f \text{ es continua} \iff \forall \, C \subset \R^m \text{ cerrado } : f^{-1} (V) \subset \R^n \text{ es 
cerrado }
}
\]
\vspace{3mm}

\underline{Demostración:} Es igual que la del Teorema 1 pero usando la premisa de que un conjunto es 
cerrado \indent si su complementario es abierto.
\vspace{7mm}

%Dieferencial, derivadas parciales y gradientes

\section{Diferencial, derivadas parciales y gradiente}
\vspace{3mm}

En $\R$ se tiene que:

\[
\boxed{
\text{Derivabilidad} \Rightarrow \text{Continuidad}
}
\]
\vspace{3mm}

%Derivadas parciales

\subsection{Derivadas parciales}
\vspace{5mm}

\definicion Sea $f : U \text{ (abierto) } \subset \R^n \to \R$, definimos la \textbf{derivada parcial i-ésima} de 
$f, \bfrac{\partial f}{\partial x_i}$, como la derivada de $f (x_1, x_2,..., x_n)$ respecto a la variable $x_i$, manteniendo fijas el resto de variables, es decir:

\[
\boxed{
\bfrac{\partial f (x_1, x_2,..., x_n)}{\partial x_i} = \displaystyle \lim_{h \to 0} \bfrac{f (x_1,..., x_i+h,...,x_n)  
- f (x_1, x_2,..., x_n)}{h}
}
\]
\vspace{5mm}

Notación en $\R^2$ y $\R^3$:
\vspace{3mm}

$\bfrac{\partial f (x, y)}{\partial x} \eqc{not} f_x (x, y), \bfrac{\partial f (x, y)}{\partial y} \eqc{not} f_y (x, y) 
$
\vspace{3mm}

$\bfrac{\partial f (x, y, z)}{\partial x} \eqc{not} f_x (x, y, z), \bfrac{\partial f (x, y, z)}{\partial y} \eqc{not} f_y (x, y, z) , \bfrac{\partial f (x, y, z)}{\partial z} \eqc{not} f_z (x, y, z)
$
\vspace{7mm}

\underline{Ejemplo:} Hallar las derivadas parciales de la función $f (x, y) = x^2 + y^4 \text{ en el punto } 
(1, 0)$.
\vspace{5mm}

\dindent $\left.\bfrac{\partial f(x, y)}{\partial x}\right |_{(1, 0)} = 2x |_{(1, 0)} = 2$
\vspace{3mm}

\dindent $\left.\bfrac{\partial f(x, y)}{\partial y}\right |_{(1, 0)} = 4y^3 |_{(1, 0)} = 0$
\vspace{7mm}

\noindent\underline{Interpretación geométrica:} $f : \R^2 \to \R$
\vspace{5mm}

Fijamos $y = y_0$ e intersecamos dicho plano con la gráfica $z = f (x, y)$ y obtenemos una curva $C (x)$.
\vspace{3mm}

$\bfrac{\partial f}{\partial x} (x_0, y_0)$ es la pendiente de la recta tangente a la curva $C (x)$ en el punto 
$(x_0, y_0)$.
\vspace{7mm}

\observacion La existencia de derivadas parciales no garantiza que el plano tangente sea "bueno", es decir, 
que sea una buena aproximación a la curva:
\vspace{5mm}

En $\R^n, n >1, \exists \text{derivadas parciales} \not\Rightarrow$ Continuidad.
\vspace{7mm}

%Diferenciación

\subsection{Diferenciación}
\vspace{7mm}

Sea $f : \R^2 \to \R$ ,para que la función sea diferenciable tiene que existirr un plano que se aproxime muy 
bien a la curva en el punto $(x_0, y_0)$, es decir:
\vspace{3mm}

$\displaystyle \lim_{(x, y) \to (x_0, y_0)} \bfrac{f (x, y) - \overbrace{\scalebox{0.7}{$(ax +by + c)$}}
^{\text{plano}}}{\| (x, y) - (x_0, y_0)\|} = 0$ ¿$a, b, c$?
\vspace{5mm}

De esta forma estamos "pidiendo" al numerador que se acerque a cero más rápido que el denominador, 
\indent es decir, que el plano se acerque a la curva más rápido de lo que lo hace hacen $x$ e $y$ al punto.
\vspace{5mm}

(1) En $(x_0, y_0) \rightarrow  z = f (x_0, y_0)$ siendo $z$ el plano y $f (x_0, y_0)$ la gráfica. $f (x_0, y_0) 
= a x_0 + b y_0 + c \Rightarrow \\ \indent \Rightarrow c = f (x_0, y_0) - a x_0 - b y_0$.
\vspace{3mm}

(2) Con $y = y_0$ la pendiente de la grafíca $f (x, y)$ en $(x_0, y_0) \text{ es } "a" \Rightarrow a =
\bfrac{\partial f (x_0, y_0)}{\partial x}$.
\vspace{5mm}

(3) De igual forma $b = \bfrac{\partial f (x_0, y_0)}{\partial y}$. (3)
\vspace{3mm}

De (1), (2), y (3) obtenemos la ecuación del plano tangente a la curva:
\vspace{3mm}

\[
\boxed{
z = f (x_0, y_0) + \bfrac{\partial f}{\partial x} (x_0, y_0) (x - x_0) + \bfrac{\partial f}{\partial y} (x_0, y_0) 
(y - y_0)
}
\]
\vspace{7mm}

Por tanto, la condición de diferenciabilidad para la función $f : \R^2 \to \R$ es:
\vspace{3mm}

%Condición de diferenciabilidad

\[
\boxed{
\displaystyle \lim_{(x, y) \to (x_0, y_0)} \bfrac{f (x, y) - f  (x_0, y_0) - \frac{\partial f}{\partial x} (x_0, y_0)
(x - x_0) - \frac{\partial f}{\partial y} (x_0, y_0) (y - y_0)}{\|(x, y) - (x_0, y_0)\|} = 0
}
\]
\vspace{3mm}

\underline{Ejemplo:} Hallar el plano tangente a la gráfica $z = x^2 + y^4$ en el punto $(1, 0, 1)$ y 
comprobar que es \\ \indent diferenciable en dicho plano.
\vspace{5mm}

\dindent $i) \, \bfrac{\partial f}{\partial x} = 2x |_{(1, 0)} = 2 \spac \bfrac{\partial f}{\partial y} = 4 y^3 |_{(1, 0)} = 
0 \Rightarrow z = 1 + 2(x -1) - 0 \cdot (y - 0) \Rightarrow  z = 2x -1$
\vspace{3mm}

\dindent $ii) \, \displaystyle \lim_{(x, y) \to (1, 0)} \bfrac{f (x, y) - f(1, 0) - \frac{\partial f}{\partial x} (1, 0) (x -1) - 
\frac{\partial f}{\partial y} (1, 0) (y - 0)}{\|(x, y) - (1, 0)\|} \eqc{\text{?}} 0$
\vspace{2mm}

\dindent $\displaystyle \lim_{(x, y) \to (1, 0)} \bfrac{(x^2 + y^4) - 1 - 2x + 2}{\sqrt{ (x - 1)^2 + y^2}} = 
\lim_{(x, y) \to (1, 0)} \bfrac{(x^2 + y^4) - (2x - 1)}{\sqrt{(x - 1)^2 + y^2}} = \lim_{(x, y) \to (1, 0)} 
\bfrac{(x - 1)^2 + y^4}{\sqrt{(x -1)^2 + y^2}}$
\vspace{5mm}

\dindent Ahora comprobamos si da 0 a partir de la definición de límite.
\vspace{3mm}

\dindent $\forall \epsilon > 0 \spac \exists \delta > 0 : \|(x - y) - (1, 0)\| < \delta \Rightarrow \left | \bfrac{(x - 1)^2 
+ y^2}{\sqrt{(x - 1)^2 + y^2}} - 0 \right | < \epsilon$
\vspace{3mm}

\dindent $\left | \bfrac{(x - 1)^2 + y^2}{\sqrt{(x - 1)^2 + y^2}} - 0 \right | \leq \bfrac{(x - 1)^2}
{\sqrt{(x - 1)^2 + y^2}} + \bfrac{y^4}{\sqrt{(x - 1)^2 + y^2}} \leq \bfrac{(x - 1)^2 + y^2}{\sqrt{(x - 
1)^2 + y^2}} + \bfrac{{[(x - 1)^2 + y^2]}^3}{\sqrt{(x - 1)^2 + y^2}} \leq \\ \dindent \leq \sqrt{(x - 1)^2 
+ y^2} + \sqrt{{[(x - 1)^2 + y^2]}^3} < \delta + \underbrace{\delta^2}_{< \delta} < 2 \delta$
\vspace{3mm}

Bastará con tomar $\delta < \sfrac{\scalebox{1.5}{$\epsilon$}}{2}$.
\vspace{7mm}

%Diferenciabilidad y matriz jacobiana

\definicion Sea $U$ un abierto en $\R^n$ decimos que $f : U  \to \R^m$ es \textbf{diferenciable} en un 
punto $x_0 = (x_0^1, x_0^2,..., x_0^n) \in U$ si existen todas las derivadas parciales de $f \text{ en } x_0$, 
y se cumple:
\vspace{3mm}

\[
\boxed{
Df(x_0) = 
\left ( \begin{array}{cccc}

\bfrac{\partial f_1}{\partial x_1} & \bfrac{\partial f_1}{\partial x_2} & \cdots & \bfrac{\partial f_1}{\partial 
x_n}\vspace{2mm}\\
\bfrac{\partial f_2}{\partial x_1} & \bfrac{\partial f_2}{\partial x_2} & \cdots & \bfrac{\partial f_2}{\partial 
x_n}\vspace{2mm}\\
\vdots & \vdots & \ddots & \vdots\vspace{2mm}\\
\bfrac{\partial f_m}{\partial x_1} & \bfrac{\partial f_m}{\partial x_2} & \cdots & \bfrac{\partial f_m}{\partial 
x_n}\\

\end{array} \right )
}
\]
\vspace{7mm}

\observacion A $Df (x_0)$, además de derivada de $f \text{ en } x_0$, también se le llama \textbf{matriz 
jacobiana} de $f \text{ en } x_0$.
\vspace{7mm}

%Gradiente

\definicion Un caso particular de $Df$ es cuando $m = 1$, es decir, el conunto de llegada de la función es 
$\R$. A esta matriz jacobiana se le llama \textbf{gradiente} y se denota:
\vspace{2mm}

\[
\boxed{
\vec{\nabla} f (x_1, x_2,..., x_n) = \left ( \bfrac{\partial f}{\partial x_1}, \bfrac{\partial f}{\partial x_2},..., 
\bfrac{\partial f}{\partial x_n}\right )
}
\]
\vspace{5mm}

%Diferenciabilidad y continuidad

\teorema Sea $U \subset \R^n$, $f : U \to \R^m$ es diferenciable en $x_0 \in U \Rightarrow f \text{ es 
continua en un entorno del punto } x_o \in U$.
\vspace{3mm}

\underline{Demostración:} $\displaystyle \lim_{x \to x_0} \|f (x) - f (x_0)\| =  \lim_{x \to x_0} 
\bfrac{\|f (x) - f (x_0) - Df (x_0) (x - x_0) + Df (x_0) (x - x_0)\| \cdot \|x - x_0\|}{\|x - x_0\|} \leqc{D. T.} \\ 
\dindent \leqc{D. T.} \lim_{x \to x_0} \cancelto{0 \text{ por ser diferenciable}}{\bfrac{\|f (x) - f (x_0) - Df 
(x_0) (x - x_0)\|}{\|x - x_0\|}} \cancelto{0}{\|x - x_0\|} + \lim_{x \to x_0} \bfrac{\|Df (x_0) (x - x_0)\|} 
{\|x - x_0\|} \|x - x_0\| = \vspace{3mm} \\ \dindent = \lim_{x - x_0} \|Df (x_0) (x - x_0)\| = \lim_{x - x_0} 
\underbrace{\|Df (x_0)\|}_{\text{es un número}} \cancelto{0}{\|x - x_0\|} = 0$
\vspace{7mm}

\teorema[1] Sea $U \subset \R^n$ un abierto y $f : U \to \R^m$:
\vspace{3mm}

\[
\boxed{
f \text{ diferenciable} \Rightarrow f \text{ continua}
}
\]
\vspace{5mm}

\teorema[2] Sea $U \subset \R^n$ un abierto, $f : U \to \R^m$ y $x_0 \in U$, si $\, \exists \bfrac{\partial f_j 
(x_0)}{\partial x_i}  \forall i =1,..., n \text{ y } \forall  j = 1,..., m$ y son continuas en un entorno de 
$x_0$, entonces $f (x)$ es diferenciable en $x_0$.
\vspace{3mm}

\underline{Ejemplo:} $f (x, y) = \left \{
\begin{array}{ccc}

\bfrac{xy^2}{\sqrt{x^2 + y^2}} & \text{si} & (x , y) \neq (0, 0)\\
0 & \text{si} & (x, y) = (0, 0)\\

\end{array}\right .
$
\vspace{3mm}

\dindent (i) Probar que es continua en todo $\R^2$ (en el origen).

\dindent (ii) Probar que es diferenciable en todo $\R^2$ (en el origen).

\dindent (iii) Probar que existen las derivadas parciales y que son continuas en $(0, 0)$ \\ \dindent (aunque 
no tienen por qué serlo).
\vspace{3mm}

\dindent (i) $\displaystyle \lim_{(x, y) \to (x_0, y_0)} \bfrac{xy^2}{\sqrt{x^2 + y^2}} \eqc{?} 0$

\dindent $\forall \epsilon > 0 \spac \exists \delta > 0 : \|(x , y) - (0, 0)\| < \delta \Rightarrow \left | 
\bfrac{xy^2}{\sqrt{x^2 + y^2}} - 0 \right | < \epsilon$
\vspace{3mm}

\dindent $\left | \bfrac{xy^2}{\sqrt{x^2 + y^2}} - 0 \right | \leqc{*} \bfrac{1}{2} 
\bfrac{\overbrace{\scalebox{0.7}{$(x^2 + y^2)$}}^{\text{plano}}|y|}{\sqrt{x^2 + y^2}} \leqc{**} 
\bfrac{1}{2} \delta^2 < \delta$. 
\vspace{3mm}

\dindent Bastará con tomar $\delta < \sqrt{2 \epsilon}$.
\vspace{3mm}

\dindent $\displaystyle \lim_{(x, y) \to (x_0, y_0)} f (x, y) = f (0, 0) = 0$
\vspace{3mm}

\dindent $\star xy \leq \bfrac{x^2 + y^2}{2}$
\vspace{2mm}

\dindent $\star \star |y| \leq \|(x, y)\| < \delta$
\vspace{5mm}

\dindent (ii) $\displaystyle \lim_{(x, y) \to (0, 0)} \bfrac{f (x, y) - f (0, 0) - f_x (0, 0) (x - 0) - f_y (0, 0) (y - 0)} 
{\|(x, y) - (0, 0)\|}$
\vspace{5mm}

\dindent $f_x (0, 0) = \bfrac{\partial f}{\partial x} (0, 0) = \displaystyle \lim_{h \to 0} \bfrac{f (h, 0) - 
f (0, 0)}{h} = \lim_{h \to 0} \bfrac{\frac{h \cdot 0^2}{\sqrt{h^2 + 0^2}} - 0}{h} = \lim_{h \to 0} 
\bfrac{0}{h^2} = 0$
\vspace{3mm}

\dindent $f_y (0, 0) = \bfrac{\partial f}{\partial y} (0, 0)= \displaystyle \lim_{h \to 0} \bfrac{f (0, h) - f (0, 0)}{h} = \lim_{h \to 0} \bfrac{\frac{0 \cdot h^2}{\sqrt{0^2 + h^2}} - 0}{h} = \lim_{h \to 0} 
\bfrac{0}{h^2} = 0$
\vspace{3mm}

\dindent$\displaystyle \lim_{(x, y) \to (0, 0)} \bfrac{\frac{xy^2}{\sqrt{x^2 + y^2}} - 0 - 0 - 0}{\sqrt{x^2 + 
y^2}} = \lim_{(x, y) \to (0, 0)} \bfrac{xy^2}{x^2 + y^2} \eqc{?} 0$
\vspace{3mm}

\dindent$\forall \epsilon >0 \spac \exists \delta > 0 : \left | \bfrac{xy^2}{x^2 + y^2} - 0 \right | \leq 
\bfrac{1}{2} \bfrac{(x^2 + y^2) |y|}{x^2 + y^2} \leq \bfrac{1}{2} \|(x, y)\| \leq \bfrac{1}{2} \delta < 
\epsilon$. \\
\vspace{3mm}

\dindent Bastará con tomar $\delta < \sfrac{\scalebox{1.5}{$\epsilon$}}{2}$.
\vspace{5mm}

\dindent (iii) $f_x = \bfrac{y^2 \sqrt{x^2 + y^2} - xy^2 \frac{x}{\sqrt{x^2 + y^2}}}{x^2 + y^2} = 
\bfrac{y^2 (x^2 + y^2) - x^2y^2}{(x^2 + y^2) \sqrt{x^2 + y^2}} = \bfrac{\cancel{{(yx)}^2} + y^4 - \cancel{{(xy)^2}}}{\sqrt{{(x^2 - y^2)}^3}} = \\ \dindent = \bfrac{y^4}{\sqrt{{(x^2 + y^2)}^3}}$
\vspace{3mm}

\dindent $\bfrac{\partial f}{\partial x} (x, y) = \left \{
\begin{array}{ccc}

\bfrac{y^4}{\sqrt{{(x^2 + y^2)}^3}} & si & (x, y) \neq (0, 0)\\
0 & si & (x, y) = (0, 0)\\

\end{array}\right .
$
\vspace{3mm}

\dindent Ahora veremos si $f_x$ es continua en el punto $(0, 0)$.
\vspace{3mm}

\dindent $\displaystyle \lim_{(x, y) \to (0, 0)} \bfrac{y^4}{\sqrt{{(x^2 + y^2)}^3}} \eqc{?} 0$
\vspace{3mm}

\dindent $\forall \epsilon > 0 \spac \exists \delta > 0 : \|(x ,y) - (0, 0)\| < \delta \Rightarrow |f (x, y) - 0| < 
\epsilon$
\vspace{3mm}

\dindent $\left | \bfrac{y^4}{\sqrt{{(x^2 + y^2)}^3}} - 0 \right | \leq \bfrac{{(x^2 + y^2)}^2} 
{\sqrt{{(x^2 + y^2)}^3}} = \sqrt{x^2 - y^2} < \delta$
\vspace{3mm}

\dindent Bastará con tomar $\delta < \epsilon$, por lo que $f_x$ es continua.
\vspace{5mm}

\dindent $f_y = \bfrac{2xy \sqrt{x^2 + y^2} - xy^2 \frac{y}{\sqrt{x^2 + y^2}}}{x^2 + y^2} = \bfrac{2xy 
(x^2 + y^2) - xy^3}{\sqrt{{(x^2 + y^2)^3}}} = \bfrac{2x^3y + 2xy^3 - xy^3}{\sqrt{{(x^2 + y^2)^3}}} 
= \\ \dindent = \bfrac{2x^3y + xy^3}{\sqrt{{(x^2 + y^2)^3}}}$
\vspace{3mm}

\dindent $\bfrac{\partial f}{\partial y} (x, y) = \left \{
\begin{array}{ccc}

\bfrac{2x^3y + xy^3}{\sqrt{{(x^2 + y^2)^3}}} & si & (x, y) \neq (0, 0)\\
0 & si & (x, y) = (0, 0)\\

\end{array}\right .
$
\vspace{3mm}

\dindent Ahora repetiremos el procedimiento con $f_y$.
\vspace{3mm}

\dindent \limdef{(0, 0)}{0}

\dindent $\left | \bfrac{2x^2y + xy^3}{\sqrt{{(x^2 + y^2)}^3}} \right | \leqc{D.T} \bfrac{2 |x^3y|} 
{\sqrt{{(x^2 + y^2)}^3}} + \bfrac{|xy^3|}{\sqrt{{(x^2 + y^2)}^3}} \leqc{*} \bfrac{(x^2 + y^2)(x^2 +
y^2)}{\sqrt{{(x^2 + y^2)}^3}} + \bfrac{1}{2} \bfrac{(x^2 + y^2)(x^2 + y^2)}{\sqrt{{(x^2 
+ y^2)}^3}} \leq \\ \dindent \leq 2 \bfrac{{x^2 + y^2)}^2}{\sqrt{{(x^2 + y^2)}^3}} \leq 2 \sqrt{(x^2 +
y^2)} < 2 \delta$
\vspace{3mm}

\dindent $ \, \star 2xy < x^2 + y^2$
\vspace{3mm}

 \dindent Bastará con tomar $\delta < \sfrac{\scalebox{1.5}{$\epsilon$}}{2}$, por lo que $f_y$ también es continua.
\vspace{7mm}

\definicion Sea $U \subset \R^n$ un abierto y $f : U \to \R^m$, decimos que $f$ es de \textbf{clase 
$C' (U)$}, y escribimos $f \in C' (U)$ si:
\vspace{3mm}
\[
\boxed{
\exists \bfrac{\partial f_i}{\partial x_j} \spac \forall i = 1, 2,..., m \text{ y } \forall j = 1, 2,..., n
}
\]
\vspace{7mm}

%Propiedades de la matriz jacobiana

\teorema Sean $U \subset \R^n$ un abierto y $f, g : U \to \R^m$ tales que $\exists Df (x_0) \text{, } Dg 
(x_0)$ matrices acobianas:
\vspace{3mm}

(1) $D (\lambda f (x_0)) = \lambda Df (x_0)$
\vspace{3mm}

(2) $D (f + g) (x_0) = Df (x_0) + Dg (x_0)$
\vspace{3mm}

(3) $D(f \cdot g) (x_0) = g (x_0) Df (x_0) + f (x_0) Dg (x_0)$
\vspace{3mm}

(4) $D (\bfrac{f}{g}) (x_0) = \bfrac{g (x_0) Df (x_0) - f (x_0) Dg (x_0)}{{[g (x_0)]}^2}$
\vspace{3mm}

(5) $D (f \circ g) (x_0) \rightarrow$ Regla de la cadena, que se explica en el apartado 2.3.
\vspace{7mm}

\subsection{Derivadas direccionales}
\vspace{5mm}

\definicion Dado un vector $\vec{u}$ unitario en $\R^n \text{, un abierto } U \subset \R^n \text{ y } \\
f : U \to \R^m$, se define la \textbf{derivada direccional} de $f$, según el vector $\vec{u}$, en un punto 
$x_0 \in U$ como:
\vspace{3mm}

\[
\boxed{
D_{\vec{u}} f (x_0) = \displaystyle \lim_{t \to 0} \bfrac{f (x_0 + t \vec{u}) - f (x_0)}{t} = \bfrac{d}{dt} 
f (x_0 + t \vec{u}) |_{t = 0}
}
\]
\vspace{5mm}

\underline{Ejemplo:}  Hallar la derivada direccional de $f : (x, y) = x^2 + y^2$ en la dirección del vector  $\vec{v} = 2 \vec{i} - 3 \vec{j}$.
\vspace{3mm}

\dindent $\vec{u} = \bfrac{\vec{v}}{\|\vec{v}} = \bfrac{2}{\sqrt{13}} \vec{i} - \bfrac{3}{\sqrt{13}} \vec{j}$
\vspace{3mm}

\dindent $D_{\vec{u}} f (1, 2) = \bfrac{d}{dt} \left .\left [f \left ( 1 + \bfrac{2t}{\sqrt{13}} , 2 - \bfrac{3t} 
{\sqrt{13}} \right ) \right ] \right |_{t =0} = \bfrac{d}{dt} \left . \left [ {\left ( 1 + \bfrac{2t}{\sqrt{13}} 
\right )}^2 + {\left ( 2 - \bfrac{3t}{\sqrt{13}} \right )}^2 \right ] \right |_{t = 0} = \\ \dindent = \bfrac{4 - 
12}{\sqrt{13}} = \bfrac{-8}{\sqrt{13}}$
\vspace{7mm}

\proposicion Sean $U \subset \R^n \text{, } f : U \to \R \text{ diferenciable  en } x_0 \in U \text{ y } \vec{u} 
$ un vector unitario, entonces:
\vspace{3mm}

\[
\boxed{
D_{\vec{u}} f (x_0) = < \vec{\nabla} f (x_0), \vec{u} >
}
\]
\vspace{3mm}

\boxed{$n =2$} \spac $f : U \subset \R^2 \to \R$
\vspace{5mm}

$D_{\vec{u}} f (x_0, y_0) = \displaystyle \lim_{t \to 0} \bfrac{f ( (x_0, y_0) + t (u_1, u_2) ) - f (x_0, y_0)}
{t} = \\ \indent = \lim_{t \to 0} \bfrac{f (x_0 - t u_1, y_0 - tu_2) - f (x_0, y_0) + f (x_0, y_0 + tu_2) - f (x_0, y_0 - tu_2)}{t} = \\ \indent = \lim_{t \to 0} \bfrac{f (x_0 + tu_1, y_0- tu_2) - f (x_0, y_0 - tu_2)}{t} + 
\lim_{t \to 0} \bfrac{f (x_0, y_0 - tu_2) - f (x_0, y_0)}{t} =$
\vspace{3mm}

$h =tu_1 \quad t \to 0 \iff h \to 0 \hspace{1.5cm} h' = tu_2 \quad t \to 0 \iff h \to 0 \hspace{1.5cm} 
\hat{y}_0 = y_0 + tu_2 \xrightarrow[t \to 0]{} y_0$
\vspace{3mm}

$=\displaystyle \underbrace{\lim_{h \to 0} \bfrac{f (x_0 + h, \hat{y}_0) - f (x_0, \hat{y}_0)}{h}} 
_{\bfrac{\partial f}{\partial x} (x_0, y_0)} u_1 + \underbrace{\lim_{h \to 0} \bfrac{f (x_0, y_0 + h') - 
f (x_0, y_0)}{h'}}_{\bfrac{\partial f}{\partial y} (x_0, y_0)} u_2$
\vspace{7mm}

\subsection{Más sobre funciones diferenciables y el vector gradiente}
\vspace{5mm}

(I) Dirección de máximo crecimiento de $f : U \subset \R^n \to \R$.
\vspace{3mm}

$f : U \subset \R^n \to \R$
\vspace{3mm}

Si $f$ es diferenciable y $\vec{u} \text{ unitario } \Rightarrow D_{\vec{u}} f (x_0) = < \vec{\nabla} f (x_0) 
, \vec{u} > = \|\vec{\nabla} f (x_0)\| \cos \theta$
\vspace{3mm}

$D_{\vec{u}} f (x_0)$ es máximo si $\cos \theta = 1 \Rightarrow \vec{\nabla} f (x_0) \text{ y } \vec{u}$ 
tienen que tener la misma dirección y el mismo sentido.
\vspace{3mm}

(i) La dirección del máximo crecimiento de la función $f$, es la del gradiente $\vec{\nabla} f$, y el valor máximo de \indent $D_{\vec{u}} f (x_0)$ es, precisamente, $\|\vec{\nabla} f (x_0)\|$. \quad $\vec{u} = 
\lambda \vec{\nabla} f (x_0) \spac (\lambda > 0)$.
\vspace{3mm}

(ii) El punto de máximo decrecimiento de la función $f$ tiene lugar cuando $\vec{\nabla} f \text{ y } \vec{u}
$ tienen la misma \indent dirección pero sentidos opuestos. $\quad \vec{u} = \lambda \vec{\nabla} f (x_0) 
\spac (\lambda < 0)$.
\vspace{3mm}

(iii) La función $f$ no varía, es decir, ni crece ni decrece, cuando $D_{\vec{u}} f (x_0) = < \vec{\nabla} f 
(x_0), \vec{u} > = 0 \Rightarrow \\ \indent \Rightarrow \vec{\nabla} f \perp \vec{u}$ y el vector gradiente 
es perpendicular a los conuntos de nivel.
\vspace{5mm}

\noindent (II) Gradiente y plano tangente a los conuntos de nivel.
\vspace{3mm}

Sea $f : \R^2 \to \R$ una función diferenciable en el punto $x_0$. Podemos considerar el plano tangente a 
la \indent curva en el punto $x_0$ de forma que se cumple que $z = f (x, y)$, y las superficies de nivel 
$F (x, y, z) = c$. \indent Entonces la ecuación del plano tangente para los conuntos de nivel será:
\vspace{3mm}

\[
\boxed{
< \vec{\nabla} F (x_0), (x - x_0, y - y_0, z - z_0) > = 0
}
\]


\end{document}